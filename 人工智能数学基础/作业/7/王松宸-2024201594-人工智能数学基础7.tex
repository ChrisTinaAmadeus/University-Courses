\documentclass[fleqn]{ctexart}
% 数学支持(提供 \mathbb、\dfrac 等)
\usepackage{amsmath,amssymb}
% 页面与版式设置:更小的页边距与顶部空白
\usepackage[a4paper,top=15mm,bottom=20mm,left=12mm,right=12mm]{geometry}
% 自动换行与字偶距优化
\usepackage{microtype}
\microtypesetup{tracking=true,protrusion=true,final}
% 图片支持
\usepackage{graphicx}
% 默认从当前文件夹查找图片
\graphicspath{{./}}
% 在需要时放宽断行以避免溢出(数值可按需调整)
\setlength{\emergencystretch}{2em}
% 列表控制,便于让(1)后直接跟内容
\usepackage{enumitem}
% 使章节标题左对齐(同时保留 ctex 默认样式)
\ctexset{section={format+=\raggedright}}
%1.3倍行距
\renewcommand{\baselinestretch}{1.3}
% 数学左对齐缩进为0
\setlength{\mathindent}{0pt}
% 增大矩阵的行距
\renewcommand{\arraystretch}{1.3}
\pagestyle{empty}
\begin{document}
\section*{第一题}
$f(x) = \frac{1}{1+ e ^{-x}} \rightarrow f'(x) = \frac{e^{-x}}{(1+e^{-x})^2}$

\section*{第二题}
$\frac{df}{dx} = \frac{ df}{d z} \cdot \frac{d z}{d y} \cdot \frac{dy}{d x}$

\medskip
其中,$\frac{df}{dz} = -\frac{1}{2} e^{-\frac{1}{2} z},\, \frac{dz}{dy} = y^T(S^{-1}+(S^{-1})^T),\, \frac{dy}{dx} = I \in \mathbb{R}^{D \times D}$

\medskip
$\therefore \frac{df}{dx} = -\frac{1}{2}e^{-\frac{1}{2} z}y^T( S^{-1}+(S^{-1})^T) = -\frac{1}{2} e^{-\frac{1}{2}((x-\mu)^TS^{-1}(x-\mu))} (x-\mu)^T( S^{-1}+(S^{-1})^T) $

\medskip
偏导数维度:

$\frac{df}{dz} \in \mathbb{R}^{1 \times 1},\, \frac{dz}{dy} \in \mathbb{R}^{1 \times D},\, \frac{dy}{dx} \in \mathbb{R}^{D \times D},$
$\frac{df}{dx} \in \mathbb{R}^{1 \times D}$

\section*{第三题}
令$x = \left(
\begin{array}{c}
x_1  \\
x_2  \\
\vdots \\
x_D
\end{array}
\right)$,则$xx^T = \left(
\begin{array}{cccc}
x_1^2 & x_1 x_2 & \cdots & x_1 x_D \\
x_2 x_1 & x_2^2 & \cdots & x_2 x_D \\
\vdots & \vdots & \ddots & \vdots \\
x_D x_1 & x_D x_2 & \cdots & x_D^2
\end{array}
\right)$

\medskip
因此,$f(x)=tr(xx^T+\sigma^2I)= \sum_{i=1}^{D} (x_i^2 + \sigma^2)$

\medskip
$\therefore \frac{df}{dx} = \left(
\begin{array}{cccc}
2x_1 & 2x_2 & \cdots & 2x_D
\end{array}
\right)$

\section*{第四题}

设$f:\mathbb{R}^n\to\mathbb{R}$在点$\mathbf{x}$可微

\medskip
取任意单位向量$\mathbf{u}$,沿$\mathbf{u}$方向的方向导数
$
 D_{\mathbf{u}}f(\mathbf{x}) 
 = \nabla f(\mathbf{x})\cdot \mathbf{u}
$

\medskip
$\because
 \nabla f(\mathbf{x})\cdot \mathbf{u}
 \le \|\nabla f(\mathbf{x})\| \cdot \|\mathbf{u}\|
 = \|\nabla f(\mathbf{x})\|
$
(当且仅当$\mathbf{u}$与$\nabla f(\mathbf{x})$共线且同向,即
$
 \mathbf{u}=\frac{\nabla f(\mathbf{x})}{\|\nabla f(\mathbf{x})\|},
$
取等号)

\medskip
$\therefore$在所有单位方向中,方向导数的最大值为$\|\nabla f(\mathbf{x})\|_2$,达到该最大值的方向就是梯度方向,故函数上升最快的方向为$\nabla f(\mathbf{x})$;同理,下降最快的方向为$-\nabla f(\mathbf{x})$

\medskip
$\therefore $ 函数变化最快的方向是梯度方向,证明完毕

\medskip
补充:若$\nabla f(\mathbf{x})=\mathbf{0}$,则对任意$\mathbf{u}$有$D_{\mathbf{u}}f(\mathbf{x})=0$,此时一阶变化对方向不敏感

\section*{第五题}
$\because f(x) = x^TAx + b^Tx +c$

\medskip
$\therefore \nabla f = x^T(A+A^T) + b^T$

\medskip
$\therefore Hessian = \nabla ^2 f = A + A^T$

\section*{第六题}
$f(x) = \langle Ax,x \rangle = (Ax)^Tx = x^T A^T x = 4x_1^2 +11x_1x_2 + 7x_2^2$

\medskip
令$x_0 = \left(
\begin{array}{c}
1 \\
1
\end{array}
\right)$

\medskip
$\therefore f(x_0)= 4·(1)^2 + 11·1·1 + 7·(1)^2 = 22$

\medskip
$\because \nabla f = \left(
\begin{array}{cc}
8x_1+11x_2 & 14x_2+11x_1
\end{array}
\right)$

\medskip
$\therefore \nabla f(x_0) = \left(
\begin{array}{cc}
19 & 25
\end{array}
\right)$

\medskip
$\because Hessian = \left(
\begin{array}{cc}
8 & 11 \\
11 & 14
\end{array}
\right)$

\medskip
$\therefore \nabla ^2 f (x_0) = \left(
\begin{array}{cc}
8 & 11 \\
11 & 14
\end{array}
\right)$

\medskip
$\therefore f(x) = f(x_0) + \nabla f(x_0)(x - x_0) + \frac{1}{2} (x-x_0)^T \nabla ^2 f(x_0) (x - x_0)$

\medskip
$ \quad \qquad \,= 22 + 19(x_1-1) + 25(x_2-1) + 4(x_1 - 1)^2 + 11(x_1-1)(x_2-1) + 7(x_2 -1)^2$
\section*{第七题}
$f(x)= e^{-\frac{1}{2 \sigma^2}(x - \mu)^2}$

\medskip
$\therefore f'(x) = e^{-\frac{1}{2 \sigma^2}(x - \mu)^2} \cdot \left(-\frac{1}{2 \sigma^2}\right) \cdot 2(x - \mu) = -\frac{1}{\sigma^2}(x - \mu) e^{-\frac{1}{2 \sigma^2}(x - \mu)^2}$
\end{document}