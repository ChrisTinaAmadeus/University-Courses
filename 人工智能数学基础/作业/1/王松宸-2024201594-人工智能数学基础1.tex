\documentclass{ctexart}
% 数学支持(提供 \mathbb、\dfrac 等)
\usepackage{amsmath,amssymb}
% 页面与版式设置:更小的页边距与顶部空白
\usepackage[a4paper,top=15mm,bottom=20mm,left=18mm,right=18mm]{geometry}
% 自动换行与字偶距优化
\usepackage{microtype}
\microtypesetup{tracking=true,protrusion=true,final}
% 在需要时放宽断行以避免溢出(数值可按需调整)
\setlength{\emergencystretch}{2em}
% 列表控制,便于让(1)后直接跟内容
\usepackage{enumitem}
% 使章节标题左对齐(同时保留 ctex 默认样式)
\ctexset{section={format+=\raggedright}}

\pagestyle{empty}
\begin{document}

\section*{第一题}
% 将小问改为行内编号而非换行标题
\begin{enumerate}[label=({\arabic*}), leftmargin=*, itemsep=0.2em, parsep=0pt, topsep=0.4em]
\item 若 $a\in \mathbb{Z}$, $b\in \mathbb{Z}$,显然有 $a-b\in \mathbb{Z}$,因此整数集 $\mathbb{Z}$ 对普通的减法运算封闭。
\item 令 $b\in \mathbb{Z}^*$,则 $\exists\, a=kb+1\in\mathbb{Z}^*\,(k\in\mathbb{Z})$ 使得 $\dfrac{a}{b}\notin\mathbb{Z}^*$,因此非零整数集 $\mathbb{Z}^*$ 对普通的除法运算不封闭。
\end{enumerate}

\section*{第二题}
% 这里写第二题内容
$\because\,a,b\,$是有限阶元

$\therefore\,$令 $\,|a|=m,|b|=n\,(m,n\in\mathbb{Z}^+)$

$\therefore\,(b^{-1}ab)^m=(b^{-1}abb^{-1}ab...b^{-1}ab)$

设 $e$ 为群 $G$ 的单位元

则$b^{-1}b=e$

$\therefore\,(b^{-1}ab)^m=b^{-1}a^mb=b^{-1}b=e$

若$\exists\,k<m(k\in\mathbb{Z}^+)$,使得$(b^{-1}ab)^k=e$

则一定有$a^k=e$,与$|a|=m$矛盾

$\therefore\,|b^{-1}ab|= m=|a|$

证明完毕
\section*{第三题}
$\because\ R=\{(1,1),(1,2),(2,1),(3,2)\}$

$\therefore\ R^2=R\circ R=\{(1,1),(1,2),(2,1),(2,2),(3,1),(3,2)\}$(新增 $(2,2),(3,1)$)

又 $\because\,R^3=R^2\circ R=\{(1,1),(1,2),(2,1),(2,2),(3,1),(3,2)\}$,无新增

$\therefore\ t(R)=\bigcup_{n\ge1}R^n=R\cup R^2=\{(1,1),(1,2),(2,1),(2,2),(3,1),(3,2)\}$

\section*{第四题}
\begin{enumerate}[label=({\arabic*}), leftmargin=*, itemsep=0.2em, parsep=0pt, topsep=0.4em]
\item 若封闭,则有$ab\in A$、$a^2\in A$等

若要满足某个数的平方仍属于集合$A$,则a,b,c的取值只能是0、1或-1

若$a,b,c$取值不为0,1,-1,则$a^2\neq a$

此时若满足封闭性则必有$b=a^2$或$c=a^2$

同理$b,c$则可推出$b=b^4$或$c=c^4$,与$a,b,c$取值不为0,1,-1矛盾

综上,可以确定$a,b,c$的取值只能是0、1或 -1使得$A$对普通乘法封闭
\item 同(1)可知若满足封闭性,则$a+b\in A$、$a+a \in A$

此时令$a=0$,则$b+b=c,$从而有$\,A=\{0,b,2b\}$

显然由$b \neq0$可知该集合对普通加法不封闭

若$a\neq0$,则同理可知该集合对普通加法依然不封闭

综上,无法确定$a,b,c$的取值使得$A$对普通加法封闭
\end{enumerate}
\end{document}