\documentclass[fleqn]{ctexart}
% 数学支持(提供 \mathbb、\dfrac 等)
\usepackage{amsmath,amssymb}
% 页面与版式设置:更小的页边距与顶部空白
\usepackage[a4paper,top=15mm,bottom=20mm,left=12mm,right=12mm]{geometry}
% 自动换行与字偶距优化
\usepackage{microtype}
\microtypesetup{tracking=true,protrusion=true,final}
% 图片支持
\usepackage{graphicx}
% 代码高亮与框线支持
\usepackage{listings}
\usepackage{xcolor}
\lstset{
    basicstyle=\ttfamily\small,
    frame=single,
    breaklines=true,
    backgroundcolor=\color{gray!5},
    keywordstyle=\color{blue},
    commentstyle=\color{gray},
    showstringspaces=false,
}
% 默认从当前文件夹查找图片
\graphicspath{{./}}
% 在需要时放宽断行以避免溢出(数值可按需调整)
\setlength{\emergencystretch}{2em}
% 列表控制,便于让(1)后直接跟内容
\usepackage{enumitem}
% 使章节标题左对齐(同时保留 ctex 默认样式)
\ctexset{section={format+=\raggedright}}
%1.3倍行距
\renewcommand{\baselinestretch}{1.3}
% 数学左对齐缩进为0
\setlength{\mathindent}{0pt}
% 增大矩阵的行距
\renewcommand{\arraystretch}{1.3}
\pagestyle{empty}
\begin{document}
\section*{第一题}
$P(X=0)=\frac{1}{3}+\frac{1}{3}=\frac{2}{3}\,,\,P(X=1)=0+\frac{1}{3}=\frac{1}{3}$

$P(Y=0)=\frac{1}{3}+0=\frac{1}{3}\,,\,P(Y=1)=\frac{1}{3}+\frac{1}{3}=\frac{2}{3}$

\begin{enumerate}[label=(\alph*), leftmargin=*, itemsep=0.2em, parsep=0pt, topsep=0.4em]
\item $H(X) = - \sum P(X = x_i) \log_{2}{P(X = x_i)} = \log_{2}{3}-\frac{2}{3}  \quad H(Y) = - \sum P(Y = y_i) \log_{2}{P(Y = y_i)} = \log_{2}{3}-\frac{2}{3} $
\item $H(X|Y) = \sum P(Y=y_i) H(X|Y=y_i)= \frac{2}{3}$

$H(Y|X) = \sum P(X=x_i) H(Y|X=x_i)=\frac{2}{3}$

\item $H(X,Y) = H(Y|X)+H(X) = \log_{2}{3}$
\item $I(X,Y)= H(X)-H(X|Y)=\log_{2}{3}-\frac{4}{3}$
\end{enumerate}

\section*{第二题}
$
0.4\,\mathcal{N}\!\left(\begin{bmatrix}10 \\\ 2\end{bmatrix},\begin{bmatrix}1 & 0\\ 0 & 1\end{bmatrix}\right)
\;+
\;0.6\,\mathcal{N}\!\left(\begin{bmatrix}0 \\\ 0\end{bmatrix},\begin{bmatrix}8.4 & 2.0\\ 2.0 & 1.7\end{bmatrix}\right)
$
\begin{enumerate}[label=(\alph*), leftmargin=*, itemsep=0.2em, parsep=0pt, topsep=0.4em]
\item 从混合的高斯分布中提取出每一维对应的均值和方差得到:

第一维$x_1$的边际分布为$0.4\,\mathcal{N}\,(10,1)+0.6\,\mathcal{N}\,(0,8.4)$

第二维$x_2$的边际分布为$0.4\,\mathcal{N}\,(2,1)+0.6\,\mathcal{N}\,(0,1.7)$
\item $\mu = 0.4 \begin{pmatrix} 10 \\ 2 \end{pmatrix} + 0.6 \begin{pmatrix} 0 \\ 0 \end{pmatrix} = \begin{pmatrix} 4 \\ 0.8 \end{pmatrix}$
\end{enumerate}
\section*{第三题}
\begin{enumerate}[label=(\alph*), leftmargin=*, itemsep=0.2em, parsep=0pt, topsep=0.4em]
\item $\because y=Ax+b+\omega$

$\therefore $在$x$固定的情况下,$Ax+b$为常数

$\because \omega \sim \mathcal{N}(\omega|0,Q)$

$\therefore p(y|x) \sim \mathcal{N}(y|Ax+b,Q)$
\item $\because x $与$\omega$相互独立

$\therefore Ax+b $与$\omega$相互独立
\[
\begin{aligned}
\therefore & \,\mu_y = E[Ax+b+\omega] = AE[x]+b+E[\omega] = A\mu + b \\
&\, \Sigma_y = Cov[y] = Cov[Ax+b+\omega] = Cov[Ax] + Cov[\omega] = ACov[x]A^T + Q = A\Sigma A^T + Q
\end{aligned}
\]
\end{enumerate}
\section*{第四题}
已知$H(x) = \int P(x) \log_{2}{P(x)}dx$,且$\mu=\int xP(x)dx$固定,$\sigma^2 =\int (x-\mu)^2P(x)dx$固定

写为优化问题:
\begin{center}
\(
\begin{aligned}
\max \quad & -\int P(x) \log_{2}{P(x)}dx \\
s.t. \quad & \int P(x) dx = 1 \\
& \int x P(x) dx = \mu \\
& \int (x-\mu)^2 P(x) dx = \sigma^2
\end{aligned}
\)
\end{center}

构造拉格朗日函数:
\begin{center}
\(
\begin{aligned}
L(P(x),\lambda_1,\lambda_2,\lambda_3) = & -\int P(x) \log_{2}{P(x)}dx + \lambda_1 \left(\int P(x) dx - 1\right) \\
& + \lambda_2 \left(\int x P(x) dx - \mu\right) + \lambda_3 \left(\int (x-\mu)^2 P(x) dx - \sigma^2\right)
\end{aligned}
\)
\end{center}

令 $L$ 对 $P(x)$ 的偏导数为 0(为了计算方便,这里取自然对数 $\ln$),来寻找使$L$最大的$P(x)$:
\begin{center}
\(
\begin{aligned}
\frac{\partial L}{\partial P(x)} = -(\ln P(x) + 1) + \lambda_1 + \lambda_2 x + \lambda_3 (x-\mu)^2 = 0
\end{aligned}
\)
\end{center}

整理得:
\begin{center}
\(
\begin{aligned}
P(x) = e^{\lambda_1 - 1 + \lambda_2 x + \lambda_3 (x-\mu)^2}
\end{aligned}
\)
\end{center}

可以看出 $P(x)$ 的指数部分是关于 $x$ 的二次函数,根据约束条件确定具体参数:

首先,由于均值被固定为 $\mu$,分布必须关于 $\mu$ 对称,这意味着 $x$ 的一次项系数 $\lambda_2$ 必须为 0。
此时表达式简化为 $P(x) = e^{\lambda_1 - 1} e^{\lambda_3 (x-\mu)^2}$。

其次,利用方差约束 $\int (x-\mu)^2 P(x) dx = \sigma^2$,可解得 $\lambda_3 = -\frac{1}{2\sigma^2}$。

最后,利用归一化约束 $\int P(x) dx = 1$,可解得常数项 $e^{\lambda_1 - 1} = \frac{1}{\sqrt{2\pi}\sigma}$。

将参数代入,最终得到:
\begin{center}
\(
\begin{aligned}
P(x) = \frac{1}{\sqrt{2\pi}\sigma} e^{-\frac{(x-\mu)^2}{2\sigma^2}}
\end{aligned}
\)
\end{center}

即高斯分布是该约束下的最大熵分布。


\section*{第五题}
设从哪个袋子中挑选水果为随机变量 $X$($X$ 可能的取值为 1, 2)。

设挑选到的水果为随机变量 $Y$(1 代表芒果,2 代表苹果),则:
P(X=1)=0.6, \quad P(X=2)=0.4

由贝叶斯公式:
\begin{center}
\(
\begin{aligned}
P(X=2 \mid Y=1) &= \frac{P(Y=1 \mid X=2)P(X=2)}{P(Y=1)} \\[0.8em]
&= \frac{P(Y=1 \mid X=2)P(X=2)}{P(Y=1 \mid X=1)P(X=1)+P(Y=1 \mid X=2)P(X=2)} \\[0.8em]
&= \frac{\dfrac{1}{2} \cdot 0.6}{\dfrac{1}{2}\cdot 0.6+\dfrac{2}{3} \cdot 0.4} \\[0.8em]
&= \frac{1}{3}
\end{aligned}
\)
\end{center}
\end{document}