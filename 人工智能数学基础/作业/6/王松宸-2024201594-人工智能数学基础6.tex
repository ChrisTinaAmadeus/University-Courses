\documentclass[fleqn]{ctexart}
% 数学支持(提供 \mathbb、\dfrac 等)
\usepackage{amsmath,amssymb}
% 页面与版式设置:更小的页边距与顶部空白
\usepackage[a4paper,top=15mm,bottom=20mm,left=12mm,right=12mm]{geometry}
% 自动换行与字偶距优化
\usepackage{microtype}
\microtypesetup{tracking=true,protrusion=true,final}
% 图片支持
\usepackage{graphicx}
% 默认从当前文件夹查找图片
\graphicspath{{./}}
% 在需要时放宽断行以避免溢出(数值可按需调整)
\setlength{\emergencystretch}{2em}
% 列表控制,便于让(1)后直接跟内容
\usepackage{enumitem}
% 使章节标题左对齐(同时保留 ctex 默认样式)
\ctexset{section={format+=\raggedright}}
%1.3倍行距
\renewcommand{\baselinestretch}{1.3}
% 数学左对齐缩进为0
\setlength{\mathindent}{0pt}
% 增大矩阵的行距
\renewcommand{\arraystretch}{1.3}
\pagestyle{empty}
\begin{document}
\section*{第一题}
$A=\left(
\begin{array}{ccc}
1 & 1 & 1 \\
1 & 5 & 5 \\
1 & 5 & 14
\end{array}
\right)$

若$A$可以进行$Cholesky$分解,则存在下三角矩阵$L=\left(
\begin{array}{ccc}
l_{11} & 0 & 0 \\
l_{21} & l_{22} & 0 \\
l_{31} & l_{32} & l_{33}
\end{array}
\right)$,使得$A=LL^{T}$

$\therefore A = LL^{T}=\left(
\begin{array}{ccc}
l_{11}^{2} & l_{11}l_{21} & l_{11}l_{31} \\
l_{11}l_{21} & l_{21}^{2}+l_{22}^{2} & l_{21}l_{31}+l_{22}l_{32} \\
l_{11}l_{31} & l_{21}l_{31}+l_{22}l_{32} & l_{31}^{2}+l_{32}^{2}+l_{33}^{2}
\end{array}
\right)$

\medskip
则$l_{11}$与$l_{21}$、$l_{31}$同号且平方为1,$l_{22}$与$l_{32}$同号且平方为4,$l_{33}^2=9$

\medskip
均取正值,解得$L=\left(
\begin{array}{ccc}
1 & 0 & 0 \\
1 & 2 & 0 \\
1 & 2 & 3
\end{array}
\right)$,故$A$可以进行$Cholesky$分解

\section*{第二题}
$A=\left(
\begin{array}{ccc}
1 & 1 & 1 \\
1 & 2 & 5 \\
1 & 5 & 3
\end{array}
\right)=\left(
\begin{array}{ccc}
a_1 & a_2 & a_3 
\end{array}
\right)
\quad \text{令}\,Q = \left(
\begin{array}{ccc}
q_1 & q_2 & q_3 
\end{array}
\right)$

\medskip
则$q_1 = \frac{a_1}{||a_1||} = \left(
\begin{array}{c}
\frac{\sqrt{3}}{3} \\
\frac{\sqrt{3}}{3}  \\
\frac{\sqrt{3}}{3}
\end{array}
\right)$
$\quad \widehat{q_2} = a_2 - (a_2 · q_1)q_1 = \left(
\begin{array}{c}
1 \\
2 \\
5
\end{array}
\right) - \frac{8\sqrt{3}}{3}\left(
\begin{array}{c}
\frac{\sqrt{3}}{3} \\
\frac{\sqrt{3}}{3} \\
\frac{\sqrt{3}}{3}
\end{array}
\right) = \left(
\begin{array}{c}
-\frac{5}{3} \\
-\frac{2}{3} \\
\frac{7}{3}
\end{array}
\right)$

\medskip
$\therefore q_2 = \frac{\widehat{q_2}}{||\widehat{q_2}||} = 
\frac{1}{\sqrt{78}} \left(
\begin{array}{c}
-5 \\
-2 \\
7
\end{array}
\right)$

\medskip
$\widehat{q_3} = a_3 - (a_3 · q_1)q_1 - (a_3 · q_2)q_2 = \frac{7}{13} \left(
\begin{array}{c}
-3 \\
4 \\
-1
\end{array}
\right)$

\medskip
$\therefore q_3 = \frac{\widehat{q_3}}{||\widehat{q_3}||} = \frac{1}{\sqrt{26}} \left(
\begin{array}{c}
-3 \\
4 \\
-1
\end{array}
\right)$

\medskip
$\therefore Q = \left(
\begin{array}{ccc}
\frac{\sqrt{3}}{3} & -\frac{5}{\sqrt{78}} & -\frac{3}{\sqrt{26}} \\
\frac{\sqrt{3}}{3} & -\frac{2}{\sqrt{78}} &  \frac{4}{\sqrt{26}} \\
\frac{\sqrt{3}}{3} &  \frac{7}{\sqrt{78}} & -\frac{1}{\sqrt{26}}
\end{array}
\right)\quad R = Q^{T}A = \left(
\begin{array}{ccc}
\sqrt{3} & \frac{8}{\sqrt{3}} & 3\sqrt{3} \\
0 & \frac{\sqrt{78}}{3} & \frac{\sqrt{78}}{13} \\
0 & 0 & \frac{7\sqrt{26}}{13}
\end{array}
\right)$

\medskip
令$A'=RQ$开始迭代直到收敛,用Python程序计算得$A' \approx \left(
\begin{array}{ccc}
7.818 & 0 & 0 \\
0 & 0.709 & 0 \\
0 & 0 & -2.526
\end{array}
\right)$

\medskip
故$A$的特征值为$7.818\,,\, 0.709\,,\, -2.526$

\section*{第三题}
$A^TA = \left(
\begin{array}{cc}
2 & -1  \\
2 & 1  \\
\end{array}
\right) \left(
\begin{array}{cc}
2 & 2  \\
-1 & 1  \\
\end{array}
\right) = \left(
\begin{array}{cc}
5 & 3 \\
3 & 5 \\
\end{array}
\right)$

\medskip
特征多项式为$|\lambda I - A^TA| = \left|
\begin{array}{cc}
\lambda - 5 & -3 \\
-3 & \lambda - 5 \\
\end{array}
\right| = (\lambda - 5)^2 - 9 = \lambda^2 - 10\lambda + 16 = (\lambda - 8)(\lambda - 2)$

\medskip
故特征值为$\lambda_1 = 8\,,\, \lambda_2 = 2 \longrightarrow D = \left(
\begin{array}{cc}
2\sqrt{2} & 0 \\
0 & \sqrt{2} \\
\end{array}
\right)$
故$A$的奇异值$\sigma_1 = 2\sqrt{2}\,,\, \sigma_2 = \sqrt{2}$

\medskip
%求特征向量
对$\lambda_1 = 8$,解方程组$(A^TA - 8I)x = 0$得特征向量$\widehat{p_1} = \left(
\begin{array}{c}
1 \\
1 \\
\end{array}
\right)$
$\longrightarrow p_1 = \frac{\widehat{p_1}}{||\widehat{p_1}||} = \frac{1}{\sqrt{2}} \left(
\begin{array}{c}
1 \\
1 \\
\end{array}
\right)$

\medskip
对$\lambda_2 = 2$,解方程组$(A^TA - 2I)x = 0$得特征向量$\widehat{p_2} = \left(
\begin{array}{c}
1 \\
-1 \\
\end{array}
\right)$
$\longrightarrow p_2 = \frac{\widehat{p_2}}{||\widehat{p_2}||} = \frac{1}{\sqrt{2}} \left(
\begin{array}{c}
1 \\
-1 \\
\end{array}
\right)$

\medskip
$\therefore P = \left(
\begin{array}{cc}
\frac{\sqrt{2}}{2} & \frac{\sqrt{2}}{2} \\
\frac{\sqrt{2}}{2} & -\frac{\sqrt{2}}{2} \\
\end{array}
\right)$

\medskip
$q_1 = \frac{1}{\sigma_1} A p_1 = \frac{1}{2\sqrt{2}} A \left(
\begin{array}{c}
\frac{\sqrt{2}}{2} \\
\frac{\sqrt{2}}{2} \\
\end{array}
\right) = \frac{1}{2\sqrt{2}} \left(
\begin{array}{c}
2\sqrt{2} \\
0 \\
\end{array}
\right) = \left(
\begin{array}{c}
1 \\
0 \\
\end{array}
\right)$

\medskip
$q_2 = \frac{1}{\sigma_2} A p_2 = \frac{1}{\sqrt{2}} A \left(
\begin{array}{c}
\frac{\sqrt{2}}{2} \\
-\frac{\sqrt{2}}{2} \\
\end{array}
\right) = \frac{1}{\sqrt{2}} \left(
\begin{array}{c}
0 \\
- \sqrt{2} \\
\end{array}
\right) = \left(
\begin{array}{c}
0 \\
-1 \\
\end{array}
\right)$

\medskip
$\therefore Q = \left(
\begin{array}{cc}
1 & 0 \\
0 & -1 \\
\end{array}
\right)$

\medskip
因此$A$的奇异值分解为$A = Q D P^{T}$

\section*{第四题}
设$A^TA$的一个特征值为$\lambda(\lambda \neq 0)$,则有$A^TAx = \lambda x$,两边左乘$A$得$AA^TAx = \lambda Ax$

\medskip
令$y = Ax$,若$y \neq 0$,则$AA^T$的一个特征值也为$\lambda$

\medskip
若$y = 0$,则$A^TAx = 0 \rightarrow \lambda =0$,矛盾

\medskip
$\therefore A^TA$的非零特征值是$AA^T$的非零特征值

\medskip
同理可证$AA^T$的非零特征值也是$A^TA$的非零特征值

\medskip
$\therefore A^TA$与$AA^T$拥有相同的非零特征值

\section*{第五题}
\begin{enumerate}[label=({\alph*}), leftmargin=*, itemsep=0.2em, parsep=0pt, topsep=0.4em]
\item 
写成矩阵形式$Ax=b$,其中
\[
A=\begin{pmatrix}4.5&3.1\\1.6&1.1\end{pmatrix},\quad b=\begin{pmatrix}19.249\\6.843\end{pmatrix},\quad \det A=4.5\cdot1.1-3.1\cdot1.6=-0.01
\]
由Cramer法则,有
\[
\begin{aligned}
x_1&=\frac{\det\big([b,\,a_2]\big)}{\det A}=\frac{19.249\cdot1.1-3.1\cdot6.843}{-0.01}=\frac{-0.0394}{-0.01}=3.94\\
x_2&=\frac{\det\big([a_1,\,b]\big)}{\det A}=\frac{4.5\cdot6.843-1.6\cdot19.249}{-0.01}=\frac{-0.0049}{-0.01}=0.49
\end{aligned}
\]

\item 
同样使用Cramer法则:
\[
\begin{aligned}
x_1&=\frac{19.249\cdot1.1-3.1\cdot6.84}{-0.01}=\frac{-0.0301}{-0.01}=3.01\\
x_2&=\frac{4.5\cdot6.84-1.6\cdot19.249}{-0.01}=\frac{-0.0184}{-0.01}=1.84
\end{aligned}
\]

\item
由于$\det A\approx-0.01$非常小,系统病态

方程右端由$6.843$到$6.84$的微小变化,引起解从$(3.94,\,0.49)$显著改变为$(3.01,\,1.84)$

为此计算$A$的条件数:

\medskip
$A^TA = \left(
\begin{array}{cc}
4.5 & 1.6 \\
3.1 & 1.1 \\
\end{array}
\right)
\left(
\begin{array}{cc}
4.5 & 3.1 \\
1.6 & 1.1 \\
\end{array}
\right) = \left(
\begin{array}{cc}
22.81 & 15.71 \\
15.71 & 10.82 \\
\end{array}
\right)$

\medskip
特征多项式为$|\lambda I - A^TA| = \left|
\begin{array}{cc}
\lambda - 22.81 & -15.71 \\
-15.71 & \lambda - 10.82 \\
\end{array}
\right| = (\lambda - 22.81)(\lambda - 10.82) - 246.8041$

\medskip
$\longrightarrow \lambda^2 - 33.63\lambda + 0.0001 = 0$

\medskip
解得特征值$\lambda_1 \approx 33.63\,,\, \lambda_2 \approx 0.000003$

\medskip
$\therefore$条件数$\kappa(A) = \sqrt{\frac{\lambda_{max}}{\lambda_{min}}} =\frac{\sigma_{max}}{\sigma_{min}}= \sqrt{\frac{33.63}{0.000003}} \approx 3348$

\medskip
过高的条件数表明该线性系统是病态的,具有高敏感性,因此输入数据的微小变化也会导致解的显著变化
\end{enumerate}
\end{document}