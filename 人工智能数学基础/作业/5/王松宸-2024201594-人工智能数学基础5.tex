\documentclass[fleqn]{ctexart}
% 数学支持(提供 \mathbb、\dfrac 等)
\usepackage{amsmath,amssymb}
% 页面与版式设置:更小的页边距与顶部空白
\usepackage[a4paper,top=15mm,bottom=20mm,left=12mm,right=12mm]{geometry}
% 自动换行与字偶距优化
\usepackage{microtype}
\microtypesetup{tracking=true,protrusion=true,final}
% 图片支持
\usepackage{graphicx}
% 默认从当前文件夹查找图片
\graphicspath{{./}}
% 在需要时放宽断行以避免溢出(数值可按需调整)
\setlength{\emergencystretch}{2em}
% 列表控制,便于让(1)后直接跟内容
\usepackage{enumitem}
% 使章节标题左对齐(同时保留 ctex 默认样式)
\ctexset{section={format+=\raggedright}}
%1.3倍行距
\renewcommand{\baselinestretch}{1.3}
% 数学左对齐缩进为0
\setlength{\mathindent}{0pt}
% 增大矩阵的行距
\renewcommand{\arraystretch}{1.3}
\pagestyle{empty}
\begin{document}
\section*{第一题}
$a=2i+3j-k,\,b=i-2j+k$

$\therefore a·b=2·1+3·(-2)+(-1)·1=-5$

$\quad a×b=(2i+3j-k)×(i-2j+k)$

\medskip
$\quad \qquad=\left(
\begin{array}{ccc}
i & j & k \\
2 & 3 & -1 \\
1 & -2 & 1
\end{array}
\right)$

\medskip
$\quad \qquad =i-3j-7k$

$\quad b×a=-(a×b)=-i+3j+7k$

$\therefore ab=-a·b+a×b=5+i-3j-7k$

\section*{第二题}
令$p=a+bi+cj+dk,\,q=e+fi+gj+hk,\, \nu=bi+cj+dk,\,\mu=fi+gj+hk$

则有$ pq=(a+ \nu)(e+\mu)=ae+a\mu+e\nu+\nu\mu=(ae-\nu · \mu)+(a\mu+e\nu+\nu × \mu)$,实数部分为$ae-\nu · \mu$

同样$qp=(e+\mu)(a+\nu)=ea+e\nu+\mu a+\mu\nu=(ea-\mu · \nu)+(e\nu+\mu a+\mu × \nu)$,实数部分为$ea-\mu · \nu$

因为内积满足交换律,故实数部分相等,证明完毕

\section*{第三题}
沿用上一题的结果有$ pq=(ae-\nu · \mu)+(a\mu+e\nu+\nu × \mu)$

$\therefore (pq)^*=(ae-\nu · \mu)-(a\mu+e\nu+\nu × \mu)$

$\because q^*=e-(fi+gj+hk)=e-\mu\, ,\,p^*=a-(bi+cj+dk)=a-\nu$

$\therefore q^*p^*=(e-\mu)(a-\nu)=ea-e\nu -a\mu +\mu \nu=(ea-\mu · \nu)-(e\nu+\mu a)+\mu × \nu=(ae-\nu · \mu)-(a\mu+e\nu+\nu × \mu)$

$\therefore (pq)^*=q^*p^*$,证明完毕

\section*{第四题}
由罗德里格斯公式$\mathbf{v}' = \mathbf{v} \cos\theta + (\mathbf{u} \times \mathbf{v}) \sin\theta + \mathbf{u}(\mathbf{u} \cdot \mathbf{v})(1 - \cos\theta)$

其中$\theta =-\frac{2}{3}\pi,\,\,u=\frac{\sqrt{3}}{3}i+\frac{\sqrt{3}}{3}j+\frac{\sqrt{3}}{3}k$

$\therefore \mathbf{v}' = \mathbf{v} \cos(-\frac{2}{3}\pi) + (\mathbf{u} \times \mathbf{v}) \sin(-\frac{2}{3}\pi) + \mathbf{u}(\mathbf{u} \cdot \mathbf{v})(1 - \cos(-\frac{2}{3}\pi))$

$\quad \quad=(-\frac{1}{2} -\frac{\sqrt{3}}{2} \mathbf{K} + \frac{3}{2}\mathbf{u} \mathbf{u}^T)\mathbf{v}$,其中$\mathbf{K}=\left(
\begin{array}{ccc}
0 & -u_z & u_y \\
u_z & 0 & -u_x \\
-u_y & u_x & 0
\end{array}
\right)$

又$\because \mathbf{u} \mathbf{u}^T=\mathbf{K}^2+\mathbf{I}$

$\therefore \mathbf{R}=\mathbf{I} -\frac{\sqrt{3}}{2} \mathbf{K} + \frac{3}{2}\mathbf{K}^2$

\medskip
$\quad \quad=\left(
\begin{array}{ccc}
1 & 0 & 0 \\
0 & 1 & 0 \\
0 & 0 & 1
\end{array}
\right)-\frac{3}{2}\left(
\begin{array}{ccc}
0 & -u_z & u_y \\
u_z & 0 & -u_x \\
-u_y & u_x & 0
\end{array}
\right)+\frac{3}{2}\left(
\begin{array}{ccc}
-\frac{2}{3} & \frac{1}{3} & \frac{1}{3} \\
\frac{1}{3} & -\frac{2}{3} & \frac{1}{3} \\
\frac{1}{3} & \frac{1}{3} & -\frac{2}{3}
\end{array}
\right)=\left(
\begin{array}{ccc}
0 & 1 & 0 \\
0 & 0 & 1 \\
1 & 0 & 0
\end{array}
\right)$

\medskip
$\because$平移向量$\mathbf{t}=\left(
\begin{array}{ccc}
2 &
3 &
4
\end{array}
\right)^T$

$\therefore$ 该变换对应的映射矩阵为$\left(
\begin{array}{cccc}
0 & 1 & 0 & 2 \\
0 & 0 & 1 & 3 \\
1 & 0 & 0 & 4 \\
0 & 0 & 0 & 1
\end{array}
\right)$

\section*{第五题}
旋转轴的单位向量$\mathbf{u}'=\left(
\begin{array}{ccc}
\frac{\sqrt{2}}{2} &\frac{\sqrt{2}}{2} &0
\end{array}
\right)^T$,旋转角度$\theta=\frac{\pi}{2}$

因此构造四元数$\mathbf{q}=\cos\frac{\pi}{4}+\sin\frac{\pi}{4}(\frac{\sqrt{2}}{2}i+\frac{\sqrt{2}}{2}j)=\frac{\sqrt{2}}{2}+\frac{1}{2}i+\frac{1}{2}j$

则$\mathbf{v}'=\mathbf{q}\mathbf{v}\mathbf{q}^{-1}$
\end{document}