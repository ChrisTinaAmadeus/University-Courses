\documentclass[fleqn]{ctexart}
% 数学支持(提供 \mathbb、\dfrac 等)
\usepackage{amsmath,amssymb}
% 页面与版式设置:更小的页边距与顶部空白
\usepackage[a4paper,top=15mm,bottom=20mm,left=12mm,right=12mm]{geometry}
% 自动换行与字偶距优化
\usepackage{microtype}
\microtypesetup{tracking=true,protrusion=true,final}
% 在需要时放宽断行以避免溢出(数值可按需调整)
\setlength{\emergencystretch}{2em}
% 列表控制,便于让(1)后直接跟内容
\usepackage{enumitem}
% 使章节标题左对齐(同时保留 ctex 默认样式)
\ctexset{section={format+=\raggedright}}
%1.3倍行距
\renewcommand{\baselinestretch}{1.3}
% 数学左对齐缩进为0
\setlength{\mathindent}{0pt}
% 增大矩阵的行距
\renewcommand{\arraystretch}{1.3}
\pagestyle{empty}
\begin{document}

\section*{第一题}
\[
A = \begin{bmatrix}
1 & -1 & 0 & 0 & 1 \\
1 & 1 & 0 & -3 & 0 \\
2 & -1 & 0 & 1 & -1 \\
-1 & 2 & 0 & -2 & -1
\end{bmatrix}
\xrightarrow{}
\begin{bmatrix}
1 & -1 & 0 & 0 & 1 \\
0 & 2 & 0 & -3 & -1 \\
0 & 1 & 0 & 1 & -3 \\
0 & 1 & 0 & -2 & 0
\end{bmatrix}
\xrightarrow{}
\begin{bmatrix}
1 & -1 & 0 & 0 & 1 \\
0 & 1 & 0 & -4 & 2 \\
0 & 0 & 0 & 1 & -1 \\
0 & 0 & 0 & 0 & 0
\end{bmatrix}
\xrightarrow{}
\begin{bmatrix}
1 & 0 & 0 & 0 & -1 \\
0 & 1 & 0 & 0 & -2 \\
0 & 0 & 0 & 1 & -1 \\
0 & 0 & 0 & 0 & 0
\end{bmatrix}
\]
\medskip
将对A所做的行变换应用到$\bar{A}$,可以求解$Ax=b$
\medskip
\[
\bar{A}=
\begin{bmatrix}
1 & -1 & 0 & 0 & 1 &3\\
1 & 1 & 0 & -3 & 0 &6\\
2 & -1 & 0 & 1 & -1 &5\\
-1 & 2 & 0 & -2 & -1 &-1
\end{bmatrix}
\xrightarrow{}
\begin{bmatrix}
1 & 0 & 0 & 0 & -1 &3\\
0 & 1 & 0 & 0 & -2 &0\\
0 & 0 & 0 & 1 & -1 &-1\\
0 & 0 & 0 & 0 & 0 &0
\end{bmatrix}
\]
\medskip
\[
\text{整理为方程组}
\left\{
\begin{array}{l}
x_1 - x_5 = 3 \\
x_2 - 2x_5 = 0 \\
x_3 = t \quad (t\in\mathbb{R})\\
x_4 - x_5 = -1 \\
x_5 = s \quad (s\in\mathbb{R})
\end{array}
\right.
\]
\medskip
\[
\text{解集}
S = \left\{\, x \in \mathbb{R}^5 \mid x =
\begin{pmatrix}
3 \\ 0 \\ 0 \\ -1 \\ 0
\end{pmatrix}
+ s
\begin{pmatrix}
1 \\ 2 \\ 0 \\ 1 \\ 1
\end{pmatrix}
+ t
\begin{pmatrix}
0 \\ 0 \\ 1 \\ 0 \\ 0
\end{pmatrix}
,\ s, t \in \mathbb{R}
\right\}
\]


\section*{第二题}
\begin{enumerate}[label=({\arabic*}), leftmargin=*, itemsep=0.2em, parsep=0pt, topsep=0.4em]
\item 将$x_1,x_2,x_3$写为矩阵$X$
\medskip
\[
X=
\begin{pmatrix}
x_1 & x_2 & x_3
\end{pmatrix}
=
\begin{pmatrix}
2 &1 &3 \\
-1 & 1 & -3 \\
3 & -2 & 8
\end{pmatrix}
\xrightarrow{}
\begin{pmatrix}
2 & 1 & 3 \\
0 & \frac{3}{2} & -\frac{3}{2} \\
0 & -\frac{7}{2} & \frac{7}{2}
\end{pmatrix}
\xrightarrow{}
\begin{pmatrix}
2 & 1 & 3 \\
0 & 1 & -1 \\
0 & 0 & 0
\end{pmatrix}
\]
\medskip
$X$的非零行有两行,即$\operatorname{rank}(X) = 2 < n$

因此,$x_1,x_2,x_3$线性相关。
\item 将$x_1,x_2,x_3$写为矩阵$X$
\medskip
\[
X=
\begin{pmatrix}
x_1 & x_2 & x_3
\end{pmatrix}
=
\begin{pmatrix}
1 &1 &1 \\
2 &1 &0 \\
1 &0 &0 \\
0 &1 &1 \\
0 &1 &1
\end{pmatrix}
\xrightarrow{}
\begin{pmatrix}
1 &1 &1 \\
0 &-1 &-2 \\
0 &0 &1 \\
0 &0 &0 \\
0 &0 &0
\end{pmatrix}
\]
\medskip
$X$的非零行有三行,即$\operatorname{rank}(X) = 3 = n$

因此,$x_1,x_2,x_3$线性无关。
\end{enumerate}
\section*{第三题}
\begin{enumerate}[label=({\arabic*}), leftmargin=*, itemsep=0.2em, parsep=0pt, topsep=0.4em]
\item 求解
$
A_{\Phi}\cdot
\begin{pmatrix}
x_1 \\ x_2 \\ x_3
\end{pmatrix}
=
\begin{pmatrix}
3x_1+2x_2+x_3\\
x_1+x_2+x_3\\
x_1-3x_2\\
2x_1+3x_2+x_3
\end{pmatrix}
$
\medskip
解得$A_{\Phi}=
\begin{pmatrix}
3 & 2 & 1 \\
1 & 1 & 1 \\
1 & -3 & 0 \\
2 & 3 & 1
\end{pmatrix}
$
\item 对$A_{\Phi}$做初等行变换
\medskip
\[
A_{\Phi}=
\begin{pmatrix}
3 & 2 & 1 \\
1 & 1 & 1 \\
1 & -3 & 0 \\
2 & 3 & 1
\end{pmatrix}
\xrightarrow{}
\begin{pmatrix}
1 & \frac{2}{3} & \frac{1}{3} \\
0 & \frac{1}{3} & \frac{2}{3} \\
0 & -\frac{11}{3} & -\frac{1}{3} \\
0 & \frac{5}{3} & -\frac{1}{3}
\end{pmatrix}
\xrightarrow{}
\begin{pmatrix}
1 & 0 & 0 \\
0 & 1 & 0 \\
0 & 0 & 1 \\
0 & 0 & 0
\end{pmatrix}
\]
\medskip
$A_{\Phi}$的非零行有三行,即$\operatorname{rank}(A_{\Phi}) = 3$
\item $\ker(A_{\Phi}) = \{\,x \in \mathbb{R}^3 \mid A_{\Phi}x = 0\,\}$

$\because\ \operatorname{rank}(A_{\Phi}) + \dim\ker(A_{\Phi}) = n = 3$

$\therefore\ \dim\ker(A_{\Phi}) = 0$

从而 $\ker(A_{\Phi}) = \{0\}$
\medskip
\[
\operatorname{Im}(A_{\Phi}) = \mathrm{span}\left\{
\begin{pmatrix}3\\1\\1\\2\end{pmatrix},\
\begin{pmatrix}2\\1\\-3\\3\end{pmatrix},\
\begin{pmatrix}1\\1\\0\\1\end{pmatrix}
\right\} \subset \mathbb{R}^4,\quad \dim\operatorname{Im}(A_{\Phi})=3.
\]
\end{enumerate}
\section*{第四题}

\textbf{($\Rightarrow$) 充分性}:

若 $Ax=b$ 有唯一解,则解存在,
故 $b\in \operatorname{Im}(A)$,从而有
$
\operatorname{rank}([A\,|\,b]) = \operatorname{rank}(A)
$

又因为解唯一,有$|A|=0
\xrightarrow{}
\dim \ker(A)=0$,由秩—零度定理 $\dim \ker(A)+\operatorname{rank}(A)=n$,得 $\operatorname{rank}(A)=n$

综上,$\operatorname{rank}([A\,|\,b])=\operatorname{rank}(A)=n$

\medskip
\textbf{($\Leftarrow$) 必要性}:

若 $\operatorname{rank}([A\,|\,b])=\operatorname{rank}(A)=n$,
则由 $\operatorname{rank}([A\,|\,b])=\operatorname{rank}(A)$ 可以确定方程组有解

其次 $\operatorname{rank}(A)=n$ 由秩—零度定理推出 $\dim\ker(A)=0$,故解至多一个

综上,解存在且至多一个,故解唯一

\medskip
\textbf{命题得证}

\section*{第五题}
若$T$为线性映射,则$T$需满足两点:

\medskip
$(1)T\begin{pmatrix} x_1+x_2 \\ y_1+y_2 \end{pmatrix}=T\begin{pmatrix} x_1 \\ y_1 \end{pmatrix}+T\begin{pmatrix} x_2 \\ y_2 \end{pmatrix}$
(2)$T\begin{pmatrix} \lambda x \\ \lambda y \end{pmatrix}=\lambda T\begin{pmatrix} x \\ y \end{pmatrix}$

\medskip
对于(1),有:

\[
T\begin{pmatrix} x_1+x_2 \\ y_1+y_2 \end{pmatrix}
=
\begin{pmatrix} 3(x_1+x_2)-(y_1+y_2) \\ y_1+y_2 \\ x_1+x_2 \end{pmatrix}
=
\begin{pmatrix} 3x_1 - y_1 \\ y_1 \\ x_1 \end{pmatrix}
\;+
\begin{pmatrix} 3x_2 - y_2 \\ y_2 \\ x_2 \end{pmatrix}
=
T\begin{pmatrix} x_1 \\ y_1 \end{pmatrix}
\;+
T\begin{pmatrix} x_2 \\ y_2 \end{pmatrix}
\]

\medskip
对于(2),有:
\[
T\begin{pmatrix} \lambda x \\ \lambda y \end{pmatrix}
=
\begin{pmatrix} 3\lambda x - \lambda y \\ \lambda y \\ \lambda x \end{pmatrix}
=
\lambda \begin{pmatrix} 3x - y \\ y \\ x \end{pmatrix}
=
\lambda T\begin{pmatrix} x \\ y \end{pmatrix}.
\]

综上,$T$为线性映射,验证完毕
\end{document}