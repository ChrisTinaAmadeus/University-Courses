\documentclass[fleqn]{ctexart}
% 数学支持(提供 \mathbb、\dfrac 等)
\usepackage{amsmath,amssymb}
% 页面与版式设置:更小的页边距与顶部空白
\usepackage[a4paper,top=15mm,bottom=20mm,left=12mm,right=12mm]{geometry}
% 自动换行与字偶距优化
\usepackage{microtype}
\microtypesetup{tracking=true,protrusion=true,final}
% 图片支持
\usepackage{graphicx}
% 默认从当前文件夹查找图片
\graphicspath{{./}}
% 在需要时放宽断行以避免溢出(数值可按需调整)
\setlength{\emergencystretch}{2em}
% 列表控制,便于让(1)后直接跟内容
\usepackage{enumitem}
% 使章节标题左对齐(同时保留 ctex 默认样式)
\ctexset{section={format+=\raggedright}}
%1.3倍行距
\renewcommand{\baselinestretch}{1.3}
% 数学左对齐缩进为0
\setlength{\mathindent}{0pt}
% 增大矩阵的行距
\renewcommand{\arraystretch}{1.3}
\pagestyle{empty}
\begin{document}
\section*{第一题}
因为函数$f$与$g$定义在相同的区间,所以设定义域为$D$,则$fg$的定义域也为$D$

$\because f$与$g$都是$D$上的凸函数

$\therefore \forall x_1, x_2 \in D, \forall \lambda \in [0,1]$,有
\begin{center}
$\displaystyle
\begin{aligned}
f(\lambda x_1 + (1-\lambda)x_2) &\leq \lambda f(x_1) + (1-\lambda)f(x_2) = A \\
g(\lambda x_1 + (1-\lambda)x_2) &\leq \lambda g(x_1) + (1-\lambda)g(x_2) = B
\end{aligned}$
\end{center}

若$h = fg$也为凸函数,则$\forall x_1, x_2 \in D, \forall \lambda \in [0,1]$,应有
\begin{center}
$\displaystyle
\begin{aligned}
h(\lambda x_1 + (1-\lambda)x_2) &\leq \lambda h(x_1) + (1-\lambda)h(x_2) \\
\Rightarrow f(\lambda x_1 + (1-\lambda)x_2)g(\lambda x_1 + (1-\lambda)x_2)
&\leq \lambda f(x_1)g(x_1) + (1-\lambda)f(x_2)g(x_2) = C
\end{aligned}$
\end{center}

比较$C$与$AB$的大小,若有$AB \leq C$则h为凸函数:

$AB - C = [\lambda f(x_1) + (1-\lambda)f(x_2)][\lambda g(x_1) + (1-\lambda)g(x_2)]-[\lambda f(x_1)g(x_1) + (1-\lambda)f(x_2)g(x_2)]$

$\qquad \qquad = \lambda (\lambda -1)(f(x_1)-f(x_2))(g(x_1)-g(x_2))$

$\because f$与$g$都非减或非增

$\therefore (f(x_1)-f(x_2))(g(x_1)-g(x_2)) \geq 0$

又$\because \lambda (\lambda -1) \leq 0$

$\therefore AB - C \leq 0 \Rightarrow AB \leq C$

$\Rightarrow h = fg$为凸函数,证毕
\section*{第二题}
将该优化问题写为标准形式:
\begin{center}
$\displaystyle
\begin{aligned}
\text{Minimize} \quad & \frac{1}{2}x^TPx+q^Tx+r \\
\text{Subject to} \quad & -1-x_i \leq 0, & i = 1,2,3\\
& x_i - 1 \leq 0, & i = 1,2,3
\end{aligned}$
\end{center}

\begin{center}
$\displaystyle
\text{其中,}  P = \left(
\begin{array}{ccc}
13 & 12 & -2 \\
12 & 17 & 6 \\
-2 & 6 & 12
\end{array}\right), \quad q = \left(
\begin{array}{c}
-22.0 \\
-14.5 \\
13.0
\end{array}\right), \quad r = 1$
\end{center}
%先验证这是一个凸问题,接着让写为Lagrange形式让梯度为0求俩参数,满足KKT条件即证明完毕

首先验证该优化问题为凸问题:

\begin{enumerate}[label=({\arabic*}), leftmargin=*, itemsep=0.2em, parsep=0pt, topsep=0.4em]
\item 对于$f_0(x)$,可以把它看作$\frac{1}{2}x^TPx$和$q^Tx+r$的加权和(权重均为1),分别验证两部分的凸性:

    a.易得仿射函数$q^Tx+r$为凸函数

    b.$\frac{1}{2}x^TPx$二阶可微,因此检验其Hessian矩阵是否半正定:

    $\quad \nabla ^2(\frac{1}{2}x^TPx) = P \rightarrow$计算$P$顺序主子式

    $\quad |P_1| = 13 > 0$

    \medskip
    $\quad |P_2| = \left|\begin{array}{cc}
13 & 12 \\
12 & 17
\end{array}\right| = 85 > 0$

\medskip
    $\quad |P_3| = |P| = 13(17 \cdot 12 - 6 \cdot 6) - 12(12 \cdot 12 - (-2) \cdot 6) + (-2)(12 \cdot 6 - 17 \cdot (-2)) = 324 > 0$

    $\quad \Rightarrow P$为正定矩阵
    
    $\quad \therefore \frac{1}{2}x^TPx$为凸函数

    $\therefore \text{结合a与b可得}f_0(x)$为凸函数

\item 对于约束条件,$f_i(x) = \pm x_i - 1$均为仿射函数,因此均为凸函数
\end{enumerate}

$\therefore$综合(1)和(2),该优化问题为凸问题

$\because$原问题是凸问题时,满足KKT条件的点即为该问题的最优解

首先,满足$-1-x^*_i \leq 0, x^*_i - 1 \leq 0$

其次,构造$Lagrange$函数:
\begin{center}
$\displaystyle
L(x,\lambda) = \frac{1}{2}x^TPx+q^Tx+r + \sum\limits_{i=1}^{3}\lambda_{1i}(-1-x_i) + \sum\limits_{i=1}^{3}\lambda_{2i}(x_{i}-1)$
\end{center}

$\therefore \nabla_x L = x^TP + q^T+(\lambda_1 - \lambda_2)^T$

\medskip
令$\lambda^*_{1i}(-1-x^*_i) = 0 \,,\,\lambda^*_{2i}(x^*_{i}-1) = 0\, ,\,\nabla_x L = 0$,代入$x^*$求解:

\medskip
解得$
\lambda_1^* = \left(
\begin{array}{c}
1 \\
0 \\
0
\end{array}\right), \quad
\lambda_2^* = \left(
\begin{array}{c}
0 \\
0 \\
2
\end{array}\right)$

\medskip
有解,且解满足$\lambda^*_i \geq 0 , i=1,2$

\medskip
$\therefore$该问题的最优解为$x^* = \left(
\begin{array}{c}
1 \\
\frac{1}{2} \\
-1
\end{array}\right)$,证毕
\section*{第三题}
\begin{enumerate}[label=({\alph*}), leftmargin=*, itemsep=0.2em, parsep=0pt, topsep=0.4em]
\item $\because (x-2)(x-4) \leq 0$

    $ \therefore $可行集为$[2,4]$

    $ \because x \in [2,4] \text{时} ,f_0(x)_{min} = 2^2 + 1 = 5$

    $ \therefore $ 最优值为$p^* = 5$,最优解为$x^* = 2$

\item 构造$Lagrange$函数:$L(x,\lambda) = x^2+1 + \lambda (x-2)(x-4) = (1 + \lambda)x^2 - 6\lambda x + (1+8\lambda)$

    绘制目标函数图像并加入$\lambda$取正值的$Lagrange$函数曲线:

    \begin{center}
    \includegraphics[width=0.6\textwidth]{image1.png}
    \end{center}

    由图像中取的正$\lambda = 0,0.5,1,2,3$,可判断对于任意$\lambda \geq 0$,都有$p^* \geq \inf\limits_{x} L(x,\lambda)$,下界性质得证
\item 对偶函数$g(\lambda) = \inf\limits_{x} L(x,\lambda) = L(\frac{3\lambda}{1+\lambda},\lambda) = \frac{1+9\lambda -\lambda ^2}{1+\lambda} (\lambda \geq 0)$

\medskip
    寻找对偶问题的最优值:$g'(\lambda) = \frac{-\lambda^2-2\lambda+8}{(1+\lambda)^2} = \frac{9}{(1+\lambda)^2} - 1$,图像为开口向下的抛物线

    $\therefore g'(\lambda) = 0 \Rightarrow \text{对偶最优解}\lambda^* = 2$

    $\therefore $对偶最优值$d^* = g(2) = 5$

    $\because d^* = p^*$

    $\therefore $强对偶性成立

    \begin{center}
    \includegraphics[width=0.6\textwidth]{image2.png}
    \end{center}
\end{enumerate}
\section*{第四题}
$L(x,\lambda,\mu) = f_0(x) + \lambda f_1(x) + \mu h_1(x) = c^Tx + \lambda^T (Gx - h) + \mu^T (Ax - b)$

$\rightarrow g(\lambda, \mu) = \inf\limits_{x} L(x,\lambda,\mu) = -\lambda^Th-\mu^Tb+\inf\limits_{x}(c^T+\lambda^T G+ \mu^T A)x$

\medskip
$\because$若线性函数不是恒值,则其下确界是$-\infty$

$\therefore$该线性规划问题的对偶函数为:

\medskip
$g(\lambda, \mu) = \left\{
\begin{array}{ll}
-\lambda^Th - \mu^Tb, & c^T + \lambda^T G + \mu^T A = 0 \\
-\infty, & \text{其他情况}
\end{array}
\right.$

\medskip
$\therefore$ 该线性规划的对偶问题为:
\begin{center}
$\displaystyle
\begin{aligned}
\text{Maximize} \quad & -\lambda^Th - \mu^Tb \\
\text{Subject to} \quad & c^T + \lambda^T G +\mu^T A = 0 \\
& \lambda \succeq 0
\end{aligned}$
\end{center}

\section*{第五题}
该问题的$KKT$条件如下:

\begin{center}
$\displaystyle
\begin{aligned}
\nabla f_0(x^*) + \nu^* \nabla h(x^*) &= 0 \\
h(x^*) &= 0 \\
\end{aligned}$
\end{center}

计算梯度:
\begin{center}
$\displaystyle
\nabla f_0(x) = \left(
\begin{array}{ccc}
-6x_1 +2 &
2x_2 + 2 &
4x_3 +2
\end{array}\right), \quad
\nabla h(x) = \left(
\begin{array}{ccc}
2x_1 &
2x_2 &
2x_3
\end{array}\right)$
\end{center}

代入$KKT$条件,得方程组:
\begin{center}
$\displaystyle
\begin{cases}
-6x_1^* + 2 + 2\nu^* x_1^* &= 0 \\
2x_2^* + 2 + 2\nu^* x_2^* &= 0 \\
4x_3^* + 2 + 2\nu^* x_3^* &= 0 \\
{x_1^*}^2 + {x_2^*}^2 +
{x_3^*}^2 - 1 &= 0
\end{cases}$
\end{center}

借助计算器得到三个解,满足 KKT 条件的所有点为:
\begin{center}
\begin{tabular}{ll}
点 1: & $x \approx \displaystyle\left(\begin{array}{c}0.1626 \\
0.4651 \\
0.8696\end{array}\right),\quad \nu \approx -3.1500$,\\
点 2: & $x \approx \displaystyle\left(\begin{array}{c}0.3604 \\
-0.8163 \\
-0.4494\end{array}\right),\quad \nu \approx 0.2250$,\\
点 3: & $x \approx \displaystyle\left(\begin{array}{c}-0.9662 \\
-0.1986 \\
-0.1657\end{array}\right),\quad \nu \approx 4.0350$.
\end{tabular}
\end{center}

求最优解,计算每一个点对应的$f_0(x)$值:
\begin{center}
\begin{tabular}{ll}
点 1: & $f_0(x) \approx -3(0.1626)^2 + (0.4651)^2 + 2(0.8696)^2 + 2(0.1626+0.4651+0.8696)$ \\
    & $\qquad\qquad\qquad\qquad\qquad\qquad\approx 4.6441$,\\
点 2: & $f_0(x) \approx -3(0.3604)^2 + (-0.8163)^2 + 2(-0.4494)^2 + 2(0.3604-0.8163-0.4494)$ \\
    & $\qquad\qquad\qquad\qquad\qquad\qquad\approx -1.1299$,\\
点 3: & $f_0(x) \approx -3(-0.9662)^2 + (-0.1986)^2 + 2(-0.1657)^2 + 2(-0.9662-0.1986-0.1657)$ \\
    & $\qquad\qquad\qquad\qquad\qquad\qquad\approx -5.3672$.
\end{tabular}
\end{center}

因此点 3 的 $f_0$ 最小,故点 3 为最优解,最优值为-5.3672,

即 $x^* \approx \displaystyle\left(\begin{array}{c}-0.9662 \\
-0.1986 \\
-0.1657\end{array}\right),\quad f_0(x^*) \approx -5.3672$.
\section*{第六题}
\begin{enumerate}[label={(\arabic*)}, leftmargin=*, itemsep=0.3em]
\item 充分性:若 $\nabla^2 f(x) \succeq 0\;(\forall x\in D)$,则对任意 $x,y\in D$ 与 $t\in[0,1]$,考虑一维函数
\[
g(t)=f\big(x+t(y-x)\big),\quad t\in[0,1].
\]
由链式法则,
\[
g'(t)=\nabla f\big(x+t(y-x)\big)^T (y-x), \qquad g''(t)=(y-x)^T \nabla^2 f\big(x+t(y-x)\big)(y-x)\ge 0.
\]
故 $g$ 在区间 $[0,1]$ 上为一维凸函数,于是
\[
g(t)\le (1-t)g(0)+t g(1)= (1-t)f(x)+t f(y).
\]
令 $t=\lambda$ 即得
\[
f\big((1-\lambda)x+\lambda y\big) \le (1-\lambda) f(x)+ \lambda f(y),\quad \forall \lambda\in[0,1].
\]
因此 $f$ 是凸函数,故充分性成立。


\item 必要性:若 $f$ 凸且 $f\in C^2(D)$,取任意点 $x\in D$ 与任意方向 $d\in\mathbb{R}^n$,考虑一维函数
\[
\phi(t)=f(x+td),\quad t\text{ 在使 }x+td\in D\text{ 的某开区间内}.
\]
由于 $f$ 凸,$\phi$ 亦凸(一维线性组合保持凸性)。

因此对一维二阶可微凸函数必有 $\phi''(0)\ge 0$。计算二阶导:
\[
\phi''(0)= d^T \nabla^2 f(x) d \ge 0,\quad \forall d\in \mathbb{R}^n.
\]
于是 $\nabla^2 f(x)$ 的二次型非负,说明 Hessian 半正定。

由于 $x$ 与 $d$ 任取,得 $\nabla^2 f(x)\succeq 0$ 对所有 $x\in D$ 成立,必要性证明完毕。
\end{enumerate}

\textbf{综上,$f$ 是凸函数 $\Longleftrightarrow$ $\nabla^2 f(x)\succeq 0$(对所有 $x\in D$),命题得证。}

\section*{第七题}
叶荫宇,斯坦福大学李国鼎讲席教授,2009年荣获INFORMS冯·诺依曼理论奖,是首位获此殊荣的华人科学家。其主要学术贡献包括:

\begin{itemize}
    \item \textbf{优化算法与复杂性理论}
    \begin{itemize}
        \item 内点法奠基:在线性规划内点法理论研究中做出开创性工作
        \item 原始-对偶预测-校正框架:显著提升大规模线性规划问题求解效率
        \item 分布式鲁棒优化:为不确定性决策提供理论工具
    \end{itemize}
    
    \item \textbf{锥优化与半定规划}
    \begin{itemize}
        \item 系统发展锥规划模型,统一并推广线性与凸优化理论
        \item 应用于传感器网络定位,提出基于半定规划的节点定位方法
    \end{itemize}
    
    \item \textbf{算法求解器与工业应用}
    \begin{itemize}
        \item 参与开发优化求解器COPT,将求解速度提升3倍以上
        \item 研究成果应用于电力系统调度、高速铁路规划等领域
    \end{itemize}
    
    \item \textbf{在线优化与学习算法}
    \begin{itemize}
        \item 推动在线线性规划发展,提出自适应决策方法
        \item 在强化学习与马尔可夫决策过程复杂度分析中做出贡献
    \end{itemize}
    
    \item \textbf{学术荣誉}
    \begin{itemize}
        \item 2009年 冯·诺依曼理论奖(INFORMS)
        \item 2012年 国际数学规划大会Tseng Lectureship奖
        \item 2014年 SIAM优化奖
        \item 谷歌学术引用超65,000次
    \end{itemize}
\end{itemize}
\end{document}