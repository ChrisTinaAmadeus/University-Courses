\documentclass[fleqn]{ctexart}
% 数学支持(提供 \mathbb、\dfrac 等)
\usepackage{amsmath,amssymb}
% 页面与版式设置:更小的页边距与顶部空白
\usepackage[a4paper,top=15mm,bottom=20mm,left=12mm,right=12mm]{geometry}
% 自动换行与字偶距优化
\usepackage{microtype}
\microtypesetup{tracking=true,protrusion=true,final}
% 图片支持
\usepackage{graphicx}
% 默认从当前文件夹查找图片
\graphicspath{{./}}
% 在需要时放宽断行以避免溢出(数值可按需调整)
\setlength{\emergencystretch}{2em}
% 列表控制,便于让(1)后直接跟内容
\usepackage{enumitem}
% 使章节标题左对齐(同时保留 ctex 默认样式)
\ctexset{section={format+=\raggedright}}
%1.3倍行距
\renewcommand{\baselinestretch}{1.3}
% 数学左对齐缩进为0
\setlength{\mathindent}{0pt}
% 增大矩阵的行距
\renewcommand{\arraystretch}{1.3}
\pagestyle{empty}
\begin{document}
\section*{第一题}
$a=1-0.1-0.3=0.6$

$b=1-0.2-0.6=0.2$

$c=1-0.8-0.1=0.1$

\medskip
$\therefore A =
\left(
\begin{array}{ccc}
0.1 & 0.2 & 0.8 \\
0.3 & 0.6 & 0.1 \\
0.6 & 0.2 & 0.1
\end{array}
\right)
$

\medskip 
设稳态向量为$\omega=\left(
\begin{array}{c}
x  \\
y  \\
z 
\end{array}
\right)$,则有$ \omega A=\omega$

\medskip
$\therefore 
\begin{cases}
0.1x+0.3y+0.6z = x \\
0.2x+0.6y+0.2z = y \\
0.8x+0.1y+0.1z = z \\
x+y+z=1
\end{cases}
$

\medskip
解得$x=y=z=\frac{1}{3}$

$\therefore$长期情况下,有雨、多云和下雨的概率均为$\frac{1}{3}$

\section*{第二题}
若$\langle\cdot,\cdot\rangle$构成内积,则需依次验证三个条件:
\begin{enumerate}[label=({\arabic*}), leftmargin=*, itemsep=0.2em, parsep=0pt, topsep=0.4em]
\item 正定性:$\langle x,x\rangle \ge 0$(当且仅当$x=0$时取等号)
\item 对称性:$\forall x,y,\ \langle x,y\rangle=\langle y,x\rangle$
\item 双线性映射:$\forall x,y,z,\text{对}\, \forall \,\alpha,\beta\in\mathbb{R},\ \langle \alpha x+\beta y,z\rangle=\alpha\langle x,z\rangle+\beta\langle y,z\rangle$且$\ \langle  x ,\alpha y+\beta z\rangle=\alpha\langle x,y\rangle+\beta\langle x,z\rangle$
\end{enumerate}

对于(1),$\forall x=(x_1,x_2)^T\neq 0$,有$x^T\left(
\begin{array}{cc}
2 & 0  \\
1 & 2  \\
\end{array}
\right)x=2x_1^2+x_1x_2+2x_2^2=x_1^2+\frac{7}{4}x_2^2+(x_1+\frac{1}{2}x_2)^2\ge 0$

当$x_1=x_2=0$时取等号,故(1)成立

对于(2),$\forall x=(x_1,x_2)^T,y=(y_1,y_2)^T$,有
$\langle x,y\rangle=x^T\left(
\begin{array}{cc}
2 & 0  \\
1 & 2  \\
\end{array}
\right)y=2x_1y_1+x_2y_1+2x_2y_2$

$=2y_1x_1+y_1x_2+2y_2x_2=y^T\left(
\begin{array}{cc}
2 & 0  \\
1 & 2  \\
\end{array}
\right)x=\langle y,x\rangle$,故成立

\medskip
对于(3),$\forall x=(x_1,x_2)^T,y=(y_1,y_2)^T,z=(z_1,z_2)^T$,

\medskip
有
$\langle \alpha x+\beta y,z\rangle=(\alpha x+\beta y)^T\left(
\begin{array}{cc}
2 & 0  \\
1 & 2  \\
\end{array}
\right)z$
$=\alpha x^T\left(
\begin{array}{cc}
2 & 0  \\
1 & 2  \\
\end{array}
\right)z+\beta y^T\left(
\begin{array}{cc}
2 & 0  \\
1 & 2  \\
\end{array}
\right)z$
$=\alpha\langle x,z\rangle+\beta\langle y,z\rangle$

\medskip
$\langle x,\alpha y+\beta z\rangle=x^T\left(
\begin{array}{cc}
2 & 0  \\
1 & 2  \\
\end{array}
\right)(\alpha y+\beta z)$
$=\alpha x^T\left(
\begin{array}{cc}
2 & 0  \\
1 & 2  \\
\end{array}
\right)y+\beta x^T\left(
\begin{array}{cc}
2 & 0  \\
1 & 2  \\
\end{array}
\right)z$
$=\alpha\langle x,y\rangle+\beta\langle x,z\rangle$,故成立

\medskip
综上,$\langle\cdot,\cdot\rangle$构成内积
\section*{第三题}
$x-y=\left(
\begin{array}{c}
2  \\
3  \\
3
\end{array}
\right)$
\medskip
\begin{enumerate}[label=({\arabic*}), leftmargin=*, itemsep=0.2em, parsep=0pt, topsep=0.4em]
\item $\langle x-y,x-y \rangle=(x-y)^T(x-y)=22$

\medskip
因此 $d(x,y)=\sqrt{\langle x-y,x-y \rangle}=\sqrt{22}$
\medskip
\item $\langle x-y,x-y \rangle=(x-y)^T\left(
\begin{array}{ccc}
2 &1 &0 \\
1 &3 &-1 \\
0 &-1 &2
\end{array}
\right)(x-y)=47$

\medskip
因此 $d(x,y)=\sqrt{\langle x-y,x-y \rangle}=\sqrt{47}$
\end{enumerate}

\section*{第四题}
\begin{enumerate}[label=({\alph*}), leftmargin=*, itemsep=0.2em, parsep=0pt, topsep=0.4em]
\item 由$U$可得$A=\left(
\begin{array}{cccc}
0 &1 &-3 &-1 \\
-1 &-3 &4 &-3 \\
2 &1 &1 &5 \\
0 &-1 &2 &0 \\
2 &2 &1 &7
\end{array}
\right)$

\medskip
$A^TA=\left(
\begin{array}{cccc}
9 & 9 & 0 & 27  \\
9 & 16 & -14 & 27  \\
0 & -14 & 31 & 3  \\
27 & 27 & 3 & 84 \\
\end{array}
\right) \qquad A^Tx=\left(
\begin{array}{c}
9 \\
23 \\
-25 \\
30 \\
\end{array}
\right)$

\medskip
$\therefore$由$A^TA \lambda =A^Tx$可得$\lambda=\left(
\begin{array}{c}
-3 \\
4 \\
1 \\
0 \\
\end{array}
\right)$

\medskip
$\therefore \pi_U(x)=A\lambda=\left(
\begin{array}{c}
1 \\
-5 \\
-1 \\
-2 \\
3
\end{array}
\right)$
\item $x-\pi_U(x)=\left(
\begin{array}{ccccc}
-2 & -4 & 0 & 6 & -2
\end{array}
\right)^T$

\medskip
$d(x,U)=||x-\pi_U(x)||=\sqrt{\langle x-\pi_U(x),x-\pi_U(x)\rangle}=2\sqrt{15}$
\end{enumerate}

\section*{第五题}
\begin{enumerate}[label=({\arabic*}), leftmargin=*, itemsep=0.2em, parsep=0pt, topsep=0.4em]
\item $A+B$是对称正定矩阵

对称性:$(A+B)^T=A^T+B^T=A+B$,故对称性成立

正定性:$\forall \,x \neq 0,\text{有} \, x^T(A+B)x=x^TAx+x^TBx>0$,故正定性成立

\medskip
\item $AB$不一定是对称正定矩阵,比如令$A=\left(
\begin{array}{cc}
1 & 1 \\
1 & 1
\end{array}
\right),\, B=\left(
\begin{array}{cc}
1 & 0 \\
0 & 2
\end{array}
\right)
$

\medskip
则$AB=\left(
\begin{array}{cc}
1 & 2 \\
1 & 2
\end{array}
\right)$,显然不对称,故$AB$不一定是对称正定矩阵

\medskip
\item $A^{-1}$为对称正定矩阵

对称性:$(A^{-1})^T=(A^T)^{-1}=A^{-1}$,故对称性成立

正定性:$\forall \,y = A^{-1}x,\text{有} \, x^TA^{-1}x=(Ay)^TA^{-1}(Ay)=y^TAy$

由于$A$正定且$y \neq 0$,故$y^TAy>0$

因此$x^TA^{-1}x>0$,故正定性成立
\end{enumerate}

\section*{第六题}
$\begin{cases}
A(0)^2 + (2)^2 + C(0)(2) + D(0) + E(2) + F &= 0 \\
A(2)^2 + (1)^2 + C(2)(1) + D(2) + E(1) + F &= 0 \\
A(1)^2 + (-1)^2 + C(1)(-1) + D(1) + E(-1) + F &= 0 \\
A(-1)^2 + (-2)^2 + C(-1)(-2) + D(-1) + E(-2) + F &= 0 \\
A(-3)^2 + (1)^2 + C(-3)(1) + D(-3) + E(1) + F &= 0 \\
A(-1)^2 + (1)^2 + C(-1)(1) + D(-1) + E(1) + F &= 0
\end{cases}$

\medskip
得到$A=\left(
\begin{array}{ccccc}
0 & 0 &0 &2 &1\\
4 & 2 & 2 & 1 & 1 \\
1 & -1 & 1 & -1 & 1 \\
1 & 2 & -1 & -2 & 1 \\
9 & -3 & -3 & 1 & 1 \\
1 & -1 & -1 & 1 & 1
\end{array}
\right),\, x=\left(
\begin{array}{c}
A \\
C \\
D \\
E \\
F
\end{array}
\right),\, b=\left(
\begin{array}{c}
-4 \\
-1 \\
-1 \\
-4 \\
-1 \\
-1
\end{array}
\right)$

\medskip
$\therefore A^TA=\left(
\begin{array}{ccccc}
100 & -19 & 20 & 11 & 16 \\
-19 & 19 & 11 & -5 & -1 \\
-20 & 11 & 16 & -1 & -2 \\
11 & -5 & -1 & 12 & 2 \\
16 & -1 & -2 & 2 & 6 \\
\end{array}
\right) \qquad A^Tb=\left(
\begin{array}{c}
-19 \\
-5 \\
5 \\
-2 \\
-12 \\
\end{array}
\right)$

\medskip
$\therefore$由$A^TA x =A^Tb$可得$x=\left(
\begin{array}{c}
\frac{189}{730} \\
-\frac{221}{365} \\
\frac{533}{730} \\
-\frac{66}{365} \\
-\frac{908}{365} \\
\end{array}
\right)$

\medskip
$\therefore$最佳拟合椭圆为$\frac{189}{730}x^2+y^2-\frac{221}{365}xy+\frac{533}{730}x-\frac{66}{365}y-\frac{908}{365}=0$

乘以730后化简得{\bfseries\boldmath 椭圆方程 $189x^2+730y^2-442xy+533x-132y-1816=0$}

在坐标系中标记出这六个点并画出椭圆

% 如果你在同一文件夹中有 image.png,可以通过下面的 figure 环境插入:
\begin{figure}[htb]
	\centering
	% width=0.8\textwidth 可以根据需要调整缩放比例
	\includegraphics[width=0.75\textwidth]{image.png}
	\caption{样本点及拟合得到的椭圆}
	\label{fig:ellipse-sample}
\end{figure}

\end{document}