\documentclass[fleqn]{ctexart}
% 数学支持(提供 \mathbb、\dfrac 等)
\usepackage{amsmath,amssymb}
% 页面与版式设置:更小的页边距与顶部空白
\usepackage[a4paper,top=15mm,bottom=20mm,left=12mm,right=12mm]{geometry}
% 自动换行与字偶距优化
\usepackage{microtype}
\microtypesetup{tracking=true,protrusion=true,final}
% 图片支持
\usepackage{graphicx}
% 代码高亮与框线支持
\usepackage{listings}
\usepackage{xcolor}
\lstset{
    basicstyle=\ttfamily\small,
    frame=single,
    breaklines=true,
    backgroundcolor=\color{gray!5},
    keywordstyle=\color{blue},
    commentstyle=\color{gray},
    showstringspaces=false,
}
% 默认从当前文件夹查找图片
\graphicspath{{./}}
% 在需要时放宽断行以避免溢出(数值可按需调整)
\setlength{\emergencystretch}{2em}
% 列表控制,便于让(1)后直接跟内容
\usepackage{enumitem}
% 使章节标题左对齐(同时保留 ctex 默认样式)
\ctexset{section={format+=\raggedright}}
%1.3倍行距
\renewcommand{\baselinestretch}{1.3}
% 数学左对齐缩进为0
\setlength{\mathindent}{0pt}
% 增大矩阵的行距
\renewcommand{\arraystretch}{1.3}
\pagestyle{empty}
\begin{document}
%$\displaystyle \lim_{n \to \infty} \frac{\log n!}{n \log n}$
\section*{第一题}
结论:$logn!$是$\Theta(n \log n)$的

证明:
\begin{enumerate}[label=({\arabic*}), leftmargin=*, itemsep=0.2em, parsep=0pt, topsep=0.4em]
\item 证明$\log n!$是$O(n \log n)$的(找上界):

\medskip
$\because$ $\log n! = \sum_{i=1}^{n} \log i \leq \sum_{i=1}^{n} \log n = n \log n$

\medskip
$\therefore$ $\log n!$是$O(n \log n)$的
\item 证明$\log n!$是$\Omega(n \log n)$的(找下界):

\medskip
$\because \log n! = \sum_{i=1}^{n} \log i \geq \sum_{i=\lceil \frac{n}{2} \rceil}^{n} \log i \geq \sum_{i=\lceil \frac{n}{2} \rceil}^{n} \log \frac{n}{2} = \frac{n}{2} \log \frac{n}{2}$

\medskip
又$\because$ $\frac{n}{2} \log \frac{n}{2} = \frac{n}{2} (\log n - 1) \geq \frac{1}{2} n · \frac{1}{2}\log n = \frac{n}{4} \log n (n \geq 4) $

\medskip
$\therefore$ $\log n!$是$\Omega(n \log n)$的

\end{enumerate}

\textbf{综上所述,$\log n!$是$\Theta(n \log n)$的}

\section*{第二题}
结论:无法推出$2^{f(x)}$是$O(2^{g(x)})$的

$\because f(x)$是$O(g(x))$的

\medskip
$\therefore$ 存在$c>0$和$x_0$,当$x \geq x_0$时,有$f(x) \leq c g(x)$

\medskip
同理若$2^{f(x)}$是$O(2^{g(x)})$的,则存在$k>0$和$x_1$,当$x \geq x_1$时,有$2^{f(x)} \leq k 2^{g(x)}$

\medskip
可以观察到指数级别的增长远快于常数级别的增长

\medskip
若取$f(x)=2x,g(x)=x$,则$f(x)$是$O(g(x))$的

\medskip
但$2^{f(x)}=2^{2x}=(2^2)^x=4^x$,而$2^{g(x)}=2^x$,$\displaystyle \lim_{x \to \infty} \frac{4^x}{2^x} = \lim_{x \to \infty} 2^x = \infty$,显然$4^x$不是$O(2^x)$的

\medskip
\textbf{综上所述,无法推出$2^{f(x)}$是$O(2^{g(x)})$的}

\section*{第三题}
\begin{enumerate}[label=({\arabic*}), leftmargin=*, itemsep=0.2em, parsep=0pt, topsep=0.4em]
\item $\because f(2)=1$

\medskip
$\therefore f(4) = 2f(2) + \log 4 = 4$

\medskip
$therefore f(16) = 2f(4) + \log 16 = 12$
\item 首先拆开几项观察规律:$f(n)=2f(n^\frac{1}{2})+\log n = 2^2f(n^\frac{1}{2^2}) + 2 \log n = 2^mf(n^\frac{1}{2^m}) + m \log n$

\medskip
因此令$n^{\frac{1}{2^m}}=2$,即$n=2^{2^m}$,则$m=\log_2 \log_2 n$

\medskip
代入上式得:$f(n) = 2^{\log_2 \log_2 n} f(2) + \log_2 \log_2 n \cdot \log n = \log_2 n + \log_2 \log_2 n \cdot \log n = \log n (\log \log n + 1 )$

\medskip
$\because \exists \,c > 0, n_0 > 0, \text{当} n \geq n_0, f(n) \leq c \log n \cdot \log \log n$

\medskip
$\therefore f(n)$是$O(\log n \cdot \log \log n)$的
\end{enumerate}

\section*{第四题}
分治算法思想:取序列中索引为$\lfloor \frac{n}{2} \rfloor$的元素,比较$a_{\lfloor \frac{n}{2}-1 \rfloor}$、$a_{\lfloor \frac{n}{2} \rfloor}$和$a_{\lfloor \frac{n}{2}+1 \rfloor}$的大小。
若$a_{\lfloor \frac{n}{2}-1 \rfloor} > a_{\lfloor \frac{n}{2} \rfloor} > a_{\lfloor \frac{n}{2}+1 \rfloor}$,则$a_m$在前半部分,进一步从$1$至$\lfloor \frac{n}{2} \rfloor$中进行递归处理;
若$a_{\lfloor \frac{n}{2}-1 \rfloor} > a_{\lfloor \frac{n}{2} \rfloor} > a_{\lfloor \frac{n}{2}+1 \rfloor}$,则$a_m$在前半部分,进一步从$1$至$\lfloor \frac{n}{2} \rfloor$中进行递归处理;
若$a_{\lfloor \frac{n}{2}-1 \rfloor} < a_{\lfloor \frac{n}{2} \rfloor} < a_{\lfloor \frac{n}{2}+1 \rfloor}$,则$a_m$在后半部分,进一步从$\lfloor \frac{n}{2} \rfloor$至$n$中进行递归处理;
若$a_{\lfloor \frac{n}{2}-1 \rfloor} < a_{\lfloor \frac{n}{2} \rfloor} $ 且$a_{\lfloor \frac{n}{2} \rfloor}> a_{\lfloor \frac{n}{2}+1 \rfloor}$,则$a_{\lfloor \frac{n}{2} \rfloor}$即为$a_m$,返回索引m

\medskip
算法Python代码:
\begin{lstlisting}[language=Python]
function findPeak(a, low, high):
    if low == high:
        return low
    mid = floor((low + high) / 2) # 取中间索引
    if mid == 1:    # 处理左侧边界
        if a[1] > a[2]:
            return 1
        else:
            return findPeak(a, 2, high)
    if mid == n:    # 处理右侧边界
        if a[n-1] < a[n]:
            return n
        else:
            return findPeak(a, low, n-1)
    # 函数核心
    if a[mid-1] < a[mid] and a[mid] > a[mid+1]:
        return mid
    elif a[mid-1] > a[mid]:  # 在递减部分,峰值在左
        return findPeak(a, low, mid-1)
    else:  # a[mid] < a[mid+1],在递增部分,峰值在右
        return findPeak(a, mid+1, high)
\end{lstlisting}

\medskip
时间复杂度分析:每次递归将问题规模减半,递归树的高度为$O(\log n)$,每层递归的工作量为$O(1)$,因此总时间复杂度为$O(\log n)$,满足分治算法的效率要求

\section*{第五题}
\begin{enumerate}[label=({\arabic*}), leftmargin=*, itemsep=0.2em, parsep=0pt, topsep=0.4em]
\item 按$Deadline$时间升序的规则排列所有程序来运行,可以最小化所有程序的最大延迟。计算按此种规则运行时这四个程序的延迟:
\begin{itemize}
\item 2:$f_2=0+t_2 =3\,,\,d_2=4 \rightarrow L = f_2 - d_2 = 3 - 4 = -1$
\item 3:$ f_3=f_2+t_3=7 \,,\, d_3=5 \rightarrow L = f_3 - d_3 = 7 - 5 = 2$
\item 1:$ f_1=f_3+t_1=9 \,,\, d_1=6 \rightarrow L = f_1 - d_1 = 9 - 6 = 3$
\item 4:$ f_4=f_1+t_4=14 \,,\, d_4=12 \rightarrow L = f_4 - d_4 = 14 - 12 = 2$
\end{itemize}
因此最大延迟为3
\item 贪心算法框架:
\begin{lstlisting}
    Sort procedures by Deadline in ascending order
    so that d1 <= d2 <= ... <= dn
    currentTime = 0
    L_max = 0
    for i from 1 to n:
        运行程序 Pi
        currentTime = currentTime + ti
        fi = currentTime
        Li = fi - di
        L_max = max(L_max, Li)
    return L_max
\end{lstlisting}

假设我们需要针对已有的调度方式进行优化,使最大延迟不断减小至最小值

那么这种情境中一定存在相邻的程序$P_i$和$P_j(i<j)$,在该调度方式下$d_i>d_j$

该调度方式下,假设$f_{i-1}=t_0$

则程序$P_i$和$P_j$的完成时间分别为$f_i=t_0+t_i$和$f_j=t_0+t_i+t_j$

此时,程序$P_i$和$P_j$的延迟分别为$L_i=f_i-d_i=t_0+t_i-d_i$和$L_j=f_j-d_j=t_0+t_i+t_j-d_j$

若交换程序$P_i$和$P_j$的调度顺序,则新的完成时间分别为$f'_j=t_0+t_j$和$f'_i=t_0+t_j+t_i$

此时,程序$P_i$和$P_j$的延迟分别为$L'_j=f'_j-d_j=t_0+t_j-d_j$和$L'_i=f'_i-d_i=t_0+t_j+t_i-d_i$

由于$d_i>d_j$,因此$t_0+t_i+t_j-d_i < t_0+t_i+t_j-d_j$,即$L'_i < L_j$

又因为$L_i=t_0+t_i-d_i < t_0+t_j-d_j=L'_j$

因此交换后顺序执行的相应位置上延迟均不大于交换前的延迟

通过不断交换相邻程序的位置可以使该调度方式的最大延迟不断减小,直至不存在相邻程序$P_i$和$P_j(i<j)$使$d_i>d_j$,即按$Deadline$升序排列
,从而证明了贪心算法的最优性
\end{enumerate}
\end{document}