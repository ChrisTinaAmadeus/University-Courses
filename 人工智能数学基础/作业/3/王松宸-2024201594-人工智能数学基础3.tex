\documentclass[fleqn]{ctexart}
% 数学支持(提供 \mathbb、\dfrac 等)
\usepackage{amsmath,amssymb}
% 页面与版式设置:更小的页边距与顶部空白
\usepackage[a4paper,top=15mm,bottom=20mm,left=12mm,right=12mm]{geometry}
% 自动换行与字偶距优化
\usepackage{microtype}
\microtypesetup{tracking=true,protrusion=true,final}
% 在需要时放宽断行以避免溢出(数值可按需调整)
\setlength{\emergencystretch}{2em}
% 列表控制,便于让(1)后直接跟内容
\usepackage{enumitem}
% 使章节标题左对齐(同时保留 ctex 默认样式)
\ctexset{section={format+=\raggedright}}
%1.3倍行距
\renewcommand{\baselinestretch}{1.3}
% 数学左对齐缩进为0
\setlength{\mathindent}{0pt}
% 增大矩阵的行距
\renewcommand{\arraystretch}{1.3}
\pagestyle{empty}
\begin{document}

\section*{第一题}
$S = \left| \det\begin{pmatrix}
3 & -4 \\
7 & -8
\end{pmatrix} \right|
=4$

\section*{第二题}
\begin{enumerate}[label=({\arabic*}), leftmargin=*, itemsep=0.2em, parsep=0pt, topsep=0.4em]
\item $\det(B)=2·\det\begin{pmatrix}
3 & 0 \\
0 & 2
\end{pmatrix}+\det\begin{pmatrix}
0 & 3 \\
1 & 0
\end{pmatrix}
=12-3=9$
\medskip
\item $
\left(
\begin{array}{ccc|ccc}
2 & 0 & 1 & 1 & 0 & 0\\
0 & 3 & 0 & 0 & 1 & 0\\
1 & 0 & 2 & 0 & 0 & 1
\end{array}
\right)
\xrightarrow{}
\left(
\begin{array}{ccc|ccc}
2 & 0 & 1 & 1 & 0 & 0\\
0 & 3 & 0 & 0 & 1 & 0\\
0 & 0 & \frac{3}{2} & -\frac{1}{2} & 0 & 1
\end{array}
\right)
\xrightarrow{}
\left(
\begin{array}{ccc|ccc}
2 & 0 & 0 & \frac{4}{3} & 0 & -\frac{2}{3}\\
0 & 1 & 0 & 0 & \frac{1}{3} & 0\\
0 & 0 & 1 & -\frac{1}{3} & 0 & \frac{2}{3}
\end{array}
\right)
\xrightarrow{}
\left(
\begin{array}{ccc|ccc}
1 & 0 & 0 & \frac{2}{3} & 0 & -\frac{1}{3}\\
0 & 1 & 0 & 0 & \frac{1}{3} & 0\\
0 & 0 & 1 & -\frac{1}{3} & 0 & \frac{2}{3}
\end{array}
\right)
$

\medskip
$\therefore B^{-1} =
\left(
\begin{array}{ccc}
\frac{2}{3} & 0 & -\frac{1}{3}\\
0 & \frac{1}{3} & 0\\
-\frac{1}{3} & 0 & \frac{2}{3}
\end{array}
\right)
$
\end{enumerate}

\section*{第三题}
本人的学号后三位为$594$,
因此初始向量$\nu_{0}=
\left(
\begin{array}{c}5\\9\\4\end{array}
\right)
$

\medskip
$
\nu_{1}=B\nu_{0}=
\left(
\begin{array}{ccc}
2 & 0 & 1\\
0 & 3 & 0\\
1 & 0 & 2
\end{array}
\right)
\left(
\begin{array}{c}
5\\
9\\
4
\end{array}
\right)
=
\left(
\begin{array}{c}
14\\
27\\
13
\end{array}
\right)
$

\medskip
$
\nu_{2}=B\nu_{1}=
\left(
\begin{array}{ccc}
2 & 0 & 1\\
0 & 3 & 0\\
1 & 0 & 2
\end{array}
\right)
\left(
\begin{array}{c}
14\\
27\\
13
\end{array}
\right)=
\left(
\begin{array}{c}
41\\
81\\
40
\end{array}
\right)
$

\medskip
$
\nu_{3}=B\nu_{2}=
\left(
\begin{array}{ccc}
2 & 0 & 1\\
0 & 3 & 0\\
1 & 0 & 2
\end{array}
\right)
\left(
\begin{array}{c}
41\\
81\\
40
\end{array}
\right)=
\left(
\begin{array}{c}
122\\
243\\
121
\end{array}
\right)
$

\medskip
可知矩阵收敛到
$
\left(
\begin{array}{c}
1\\
2\\
1
\end{array}
\right)
$,
即占优特征向量为
$
\left(
\begin{array}{c}
1\\
2\\
1
\end{array}
\right)
$

\medskip
由$(A-\lambda I)
\left(
\begin{array}{c}
1\\
2\\
1
\end{array}
\right)=0$可求出占优特征值$\lambda=3$
\section*{第四题}
\begin{enumerate}[label=({\arabic*}), leftmargin=*, itemsep=0.2em, parsep=0pt, topsep=0.4em]
\item $
(A-\lambda I)\xi = 0
$

\medskip
$\xrightarrow{}
\left(
\begin{array}{ccc}
2-\lambda & -1 & 2\\
5 & a-\lambda & 3\\
-1 & b & -2-\lambda
\end{array}
\right)
·
\left(
\begin{array}{c}
1\\
1 \\
-1
\end{array}
\right)
=0
$

\medskip
$
\xrightarrow{}
\left(
\begin{array}{c}
-1-\lambda\\2+a+\lambda\\b+1+\lambda
\end{array}
\right)
=\left(
\begin{array}{c}
0\\0\\0
\end{array}
\right)
$

\medskip
$\therefore \lambda=-1,a=-3,b=0$

\item $
\det(A-\lambda I)
=\det\left(
\begin{array}{ccc}
2-\lambda & -1 & 2\\
5 & -3-\lambda & 3\\
-1 & 0 & -2-\lambda
\end{array}
\right)
=-(\lambda+1)^3
$

\medskip
$\therefore \lambda=-1$为A的唯一特征值,代数重数为3

$ A+\lambda I =
\left(
\begin{array}{ccc}
3 & -1 & 2\\
5 & -2 & 3\\
-1 & 0 & -1
\end{array}
\right)
\xrightarrow{}
\left(
\begin{array}{ccc}
1 & 0 & 1\\
0 & 1 & 1\\
0 & 0 & 0
\end{array}
\right)
$

\medskip
$
\therefore rank(A+\lambda I)=2 \xrightarrow{} \text{几何重数为}3-2=1
$

$\therefore$ 代数重数不等于几何重数,故$A$不与对角矩阵相似
\end{enumerate}
\section*{第五题}
\begin{enumerate}[label=({\arabic*}), leftmargin=*, itemsep=0.2em, parsep=0pt, topsep=0.4em]
\item $
\det(A-\lambda I)=\det
\left(
\begin{array}{cccc}
4-\lambda & -9 & 6 & 12\\
0 & -1-\lambda & 4 & 6\\
2 & -11 & 8-\lambda & 16\\
-1 & 3 & 0 & -1-\lambda
\end{array}
\right)
=\lambda^4 - 10\lambda^3 + 35\lambda^2 - 50\lambda + 24
=(\lambda-1)(\lambda-2)(\lambda-3)(\lambda-4)
$

$\lambda_1=1$时,$A-\lambda_1 I=
\left(
\begin{array}{cccc}
3 & -9 & 6 & 12\\
0 & -2 & 4 & 6\\
2 & -11 & 7 & 16\\
-1 & 3 & 0 & -2
\end{array}
\right)
\xrightarrow{}
\left(
\begin{array}{cccc}
1 & 0 & 0 & -1\\
0 & 1 & 0 & -1\\
0 & 0 & 1 & 1\\
0 & 0 & 0 & 0
\end{array}
\right)
$

\medskip
$\therefore$ $\lambda_1$的一个特征向量$\nu_{1}$为
$\left(
\begin{array}{cccc}1&1&-1&1\end{array}
\right)^T$

\medskip
同理可以得到$\lambda_2$的一个特征向量$\nu_{2}$为
$\left(
\begin{array}{cccc}3&2&0&1\end{array}
\right)^T$, 

\medskip
$\lambda_3$的一个特征向量$\nu_{3}$为
$\left(
\begin{array}{cccc}3&1&1&0\end{array}
\right)^T$, 

\medskip
$\lambda_4$的一个特征向量$\nu_{4}$为
$\left(
\begin{array}{cccc}1&2&1&1\end{array}
\right)^T$

因此$A$的对角化矩阵$B$为
$
\left(
\begin{array}{cccc}
1 & 0 & 0 & 0\\
0 & 2 & 0 & 0\\
0 & 0 & 3 & 0\\
0 & 0 & 0 & 4
\end{array}
\right)
$,特征向量组成的矩阵$P$为
$
\left(
\begin{array}{cccc}
1 & 3 & 3 & 1\\
1 & 2 & 1 & 2\\ 
-1 & 0 & 1 & 1\\
1 & 1 & 0 & 1
\end{array}
\right)
$

矩阵$P$的逆矩阵$P^{-1}$为
$\left(
\begin{array}{cccc}
1 & -5 & 2 & 7\\
-1 & 6 & -3 & -8\\
1 & -4 & 2 & 5\\
0 & -1 & 1 & 2
\end{array}
\right)$

$A=PBP^{-1}$验证成立
\medskip
\item $A^{10}=P B^{10} P^{-1}$

\medskip
$\because
B^{10}=
\left(
\begin{array}{cccc}
1^{10} & 0 & 0 & 0\\
0 & 2^{10} & 0 & 0\\
0 & 0 & 3^{10} & 0\\
0 & 0 & 0 & 4^{10}
\end{array}
\right)
=\left(
\begin{array}{cccc}
1 & 0 & 0 & 0\\
0 & 1024 & 0 & 0\\
0 & 0 & 59049 & 0\\
0 & 0 & 0 & 1048576
\end{array}
\right)
$

\medskip
$\therefore $使用计算器计算得到$A^{10}=P B^{10} P^{-1}=
\left(
\begin{array}{cccc}
174076 & -1738737 & 1393656 & 2958318 \\
57002 & -2321065 & 2209108 & 4473172 \\
59048 & -1284767 & 1166672 & 2392390 \\
-1023 & -1042437 & 1045506 & 2088967
\end{array}
\right)
$

\end{enumerate}
\end{document}