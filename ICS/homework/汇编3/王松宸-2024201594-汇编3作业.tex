\documentclass[fleqn]{ctexart}
\usepackage{graphicx}
% 数学支持(提供 \mathbb、\dfrac 等)
\usepackage{amsmath,amssymb}
% 页面与版式设置:更小的页边距与顶部空白
\usepackage[a4paper,top=15mm,bottom=20mm,left=12mm,right=12mm]{geometry}
% 自动换行与字偶距优化
\usepackage{microtype}
\microtypesetup{tracking=true,protrusion=true,final}
% 在需要时放宽断行以避免溢出(数值可按需调整)
\setlength{\emergencystretch}{2em}
% 列表控制,便于让(1)后直接跟内容
\usepackage{enumitem}
% 使章节标题左对齐(同时保留 ctex 默认样式)
\ctexset{section={format+=\raggedright}}
%1.3倍行距
\renewcommand{\baselinestretch}{1.3}
% 数学左对齐缩进为0
\setlength{\mathindent}{0pt}
% 增大矩阵的行距
\renewcommand{\arraystretch}{1.3}
\pagestyle{empty}
\begin{document}
\section*{题目1}
阅读汇编代码,根据唯一的减法可以得到x与c分别被放置在edx和eax寄存器里,从而可以由指令后的l与swl得知它们的大小与
所在地址。接着根据改变内存的这一行可以得到p和d的大小与所在地址。
\begin{table}[htbp]
	\centering
	\begin{tabular}{|c|c|c|}
		\hline
        参数 & 大小(字节) & 地址 \\
        \hline
        c &2& ebp+8\\
        \hline
        x &4& ebp+20 \\
        \hline
        d &1& ebp+12 \\
        \hline
        p &4& ebp+16 \\
		\hline
	\end{tabular}
\end{table}

因为参数从右向左入栈,所以函数原型为func(int x, char *p, char d, short c)
\section*{题目2}
(1)
\begin{table}[htbp]
	\centering
	\begin{tabular}{|c|c|}
		\hline
        寄存器 & value \\
        \hline
        esp & 0x7FFFFFC0\\
        \hline
        ebp &0x7FFFFFF4\\
        \hline
	\end{tabular}
\end{table}

(2)
\begin{table}[htbp]
	\centering
	\begin{tabular}{|c|c|}
		\hline
        寄存器 & value \\
        \hline
        esp & 0x7FFFFFC0\\
        \hline
        ebp &0x7FFFFFC0\\
        \hline
	\end{tabular}
\end{table}

(3)
\begin{center}
	\begin{tabular}{|c|c|}
		\hline
        寄存器 & value \\
        \hline
        esp & 0x7FFFFFC4\\
        \hline
        ebp &0x7FFFFFC4\\
        \hline
	\end{tabular}
\end{center}

\section*{题目3}
\begin{verbatim}
int main{
    int a,b;
    scanf("%d %d",&a,&b);
    int c = a ^ b;
    printf("%d %d %d\n",c,b,a);
    return 0;
}
\end{verbatim}

\begin{center}
    \includegraphics[width=0.8\textwidth]{image.jpg}
\end{center}

\end{document}