\documentclass[fleqn]{ctexart}
% 数学支持(提供 \mathbb、\dfrac 等)
\usepackage{amsmath,amssymb}
% 页面与版式设置:更小的页边距与顶部空白
\usepackage[a4paper,top=15mm,bottom=20mm,left=12mm,right=12mm]{geometry}
% 自动换行与字偶距优化
\usepackage{microtype}
\microtypesetup{tracking=true,protrusion=true,final}
% 在需要时放宽断行以避免溢出(数值可按需调整)
\setlength{\emergencystretch}{2em}
% 列表控制,便于让(1)后直接跟内容
\usepackage{enumitem}
% 使章节标题左对齐(同时保留 ctex 默认样式)
\ctexset{section={format+=\raggedright}}
%1.3倍行距
\renewcommand{\baselinestretch}{1.3}
% 数学左对齐缩进为0
\setlength{\mathindent}{0pt}
% 增大矩阵的行距
\renewcommand{\arraystretch}{1.3}
\pagestyle{empty}
\begin{document}
\section*{题目1}

\begin{table}[htbp]
	\centering
	\begin{tabular}{|c|c|}
		\hline
        Value & Two's Complement \\
        \hline
        37 & 00100101 \\
        \hline
        -15 & 11110001 \\
        \hline
        85 & 01010101 \\
        \hline
        -86 & 10101010 \\
		\hline
	\end{tabular}
\end{table}

\section*{题目2}
\begin{table}[htbp]
	\centering
	\begin{tabular}{|c|c|}
		\hline
        Expression & Binary Representation \\
        \hline
        us & 1101 \\
        \hline
        ui & 11001100 \\
        \hline
        us << 1 & 1010 \\
        \hline
        i >> 2 & 11110011 \\
		\hline
        ui >> 2 & 00110011 \\
		\hline
        (short)i & 1100 \\
		\hline
        (int)s & 11111101 \\
		\hline
	\end{tabular}
\end{table}

\section*{题目3}
%添加一段代码
\begin{verbatim}
int uadd_ok(unsigned x, unsigned y){
    unsigned sum = x + y;
    return sum < x;
}
\end{verbatim}

无符号加法发生溢出的条件是结果小于任一加数。
\end{document}