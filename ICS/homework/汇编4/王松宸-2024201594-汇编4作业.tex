\documentclass[fleqn]{ctexart}
\usepackage{graphicx}
% 数学支持(提供 \mathbb、\dfrac 等)
\usepackage{amsmath,amssymb}
% 页面与版式设置:更小的页边距与顶部空白
\usepackage[a4paper,top=15mm,bottom=20mm,left=12mm,right=12mm]{geometry}
% 自动换行与字偶距优化
\usepackage{microtype}
\microtypesetup{tracking=true,protrusion=true,final}
% 在需要时放宽断行以避免溢出(数值可按需调整)
\setlength{\emergencystretch}{2em}
% 列表控制,便于让(1)后直接跟内容
\usepackage{enumitem}
% 使章节标题左对齐(同时保留 ctex 默认样式)
\ctexset{section={format+=\raggedright}}
%1.3倍行距
\renewcommand{\baselinestretch}{1.3}
% 数学左对齐缩进为0
\setlength{\mathindent}{0pt}
% 增大矩阵的行距
\renewcommand{\arraystretch}{1.3}
\pagestyle{empty}
\begin{document}
\section*{题目1}

结构体成员偏移量分析:
\begin{itemize}
\item \texttt{a}: 偏移0,占1字节,补齐1字节
\item \texttt{b}: 偏移2,占4字节,补齐2字节
\item \texttt{c}: 偏移8,占4字节
\item \texttt{p}: 偏移12,占4字节
\item \texttt{d}: 偏移16,占1字节,补齐3字节
\end{itemize}
结构体总大小为20字节(0x14)

\begin{table}[htbp]
	\centering
	\begin{tabular}{|c|c|c|}
		\hline
        Variable & Start address \\
        \hline
        d[0] & 0x8049600\\
        \hline
        d[1] & 0x8049614 \\
        \hline
        d[0].a & 0x8049600 \\
        \hline
        d[0].b[1] & 0x8049604 \\
		\hline
        d[0].c & 0x8049608 \\
		\hline
        d[0].p.y & 0x804960C \\
		\hline
        d[0].p.z & 0x804960C \\
		\hline
        d[0].d & 0x8049610 \\
		\hline
	\end{tabular}
\end{table}

\section*{题目2}
do you want a midterm exam?

yes!


第一句话要注意“\textbackslash 0”到底在哪里

第二句话,char **按指针(大小为4个字节)移动,char *和ans的表意一致,直接找索引即可

\section*{题目3}
\begin{table}[htbp]
\centering
\begin{tabular}{|c|c|c|c|c|c|c|}
\hline
& \multicolumn{4}{c|}{Offset of each field} & Total size & Alignment \\
\hline
A & i:0 & c:4 & j:8 & d:16 & 24 & 8 \\
\hline
B & l:0 & c:8 & d:9 & j:12 & 16 & 8 \\
\hline
C & \multicolumn{2}{c|}{w:0} & \multicolumn{2}{c|}{c:8}  & 32 & 8 \\
\hline
D & \multicolumn{2}{c|}{a:0} & \multicolumn{2}{c|}{p:48}  & 56 & 8 \\
\hline
E & \multicolumn{2}{c|}{w:0} & \multicolumn{2}{c|}{c:6}  & 10 & 2 \\
\hline
\end{tabular}
\end{table}

\section*{题目4}
\begin{table}[htbp]
	\centering
	\begin{tabular}{|c|}
		\hline
        ret add \\
        \hline
        ebp \\
        \hline
           \\
        \hline
        username  \\
        \hline
        password  \\
		\hline
	\end{tabular}
\end{table}

根据栈帧,username前20字节可以输入任意值,之后的4个字节需要依照小端序输入0xda、0x13、0x40、0x80对应的字符
\end{document}