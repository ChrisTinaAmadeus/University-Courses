\documentclass[fleqn]{ctexart}
% 数学支持(提供 \mathbb、\dfrac 等)
\usepackage{amsmath,amssymb}
% 页面与版式设置:更小的页边距与顶部空白
\usepackage[a4paper,top=15mm,bottom=20mm,left=12mm,right=12mm]{geometry}
% 自动换行与字偶距优化
\usepackage{microtype}
\microtypesetup{tracking=true,protrusion=true,final}
% 在需要时放宽断行以避免溢出(数值可按需调整)
\setlength{\emergencystretch}{2em}
% 列表控制,便于让(1)后直接跟内容
\usepackage{enumitem}
% 使章节标题左对齐(同时保留 ctex 默认样式)
\ctexset{section={format+=\raggedright}}
%1.3倍行距
\renewcommand{\baselinestretch}{1.3}
% 数学左对齐缩进为0
\setlength{\mathindent}{0pt}
% 增大矩阵的行距
\renewcommand{\arraystretch}{1.3}
\pagestyle{empty}
\begin{document}
\section*{题目1}
%制作一个6×4的表格
\begin{table}[htbp]
	\centering
	\begin{tabular}{|c|c|c|c|}
		\hline
		Binary & Octal & Decimal & Hexadecimal \\
		\hline
		101 0101 0110 & 2526 & 1366 & 0x556 \\
		\hline
		1 1111 1111 & 777 & 511 & 0x1FF \\
		\hline
		1 1100 0101 & 705 & 453 & 0x1C5 \\
		\hline
		111 1101 1111 & 3737 & 2015 & 0x7DF \\
		\hline
		100 0000 1101 & 2015 & 1037 & 0x40D \\
		\hline
	\end{tabular}
\end{table}

\section*{题目2}
将$A$与$B$转化为二进制形式,得到$01111111$和$10111010$
\begin{enumerate}[label=({\alph*}), leftmargin=*, itemsep=0.2em, parsep=0pt, topsep=0.4em]
\item $A\&B=00111010=0x3A$
\item $A|B=11111111=0xFF$
\item $A\mathbin{\char`\^}B=11000101=0xC5$
\item $\sim A|\sim B=10000000|01000101=11000101=0xC5$
\item $A\&\&B=true$
\item $A||B=true$
\end{enumerate}

\section*{题目3}
\begin{enumerate}[label=({\alph*}), leftmargin=*, itemsep=0.2em, parsep=0pt, topsep=0.4em]
\item ! ($\sim x$)
\item ! $x$
\item ! ($(x \& 0xFF)\mathbin{\char`\^}0xFF$)
\item ! $(x \& 0xFF)$
\end{enumerate}

\section*{题目4}
$(x|0xFFFF0000) \& (y|0xFFFF)$
\end{document}