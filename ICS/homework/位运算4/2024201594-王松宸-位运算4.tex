\documentclass[fleqn]{ctexart}
% 数学支持(提供 \mathbb、\dfrac 等)
\usepackage{amsmath,amssymb}
% 页面与版式设置:更小的页边距与顶部空白
\usepackage[a4paper,top=15mm,bottom=20mm,left=12mm,right=12mm]{geometry}
% 自动换行与字偶距优化
\usepackage{microtype}
\microtypesetup{tracking=true,protrusion=true,final}
% 在需要时放宽断行以避免溢出(数值可按需调整)
\setlength{\emergencystretch}{2em}
% 列表控制,便于让(1)后直接跟内容
\usepackage{enumitem}
% 使章节标题左对齐(同时保留 ctex 默认样式)
\ctexset{section={format+=\raggedright}}
%1.3倍行距
\renewcommand{\baselinestretch}{1.3}
% 数学左对齐缩进为0
\setlength{\mathindent}{0pt}
% 增大矩阵的行距
\renewcommand{\arraystretch}{1.3}
\pagestyle{empty}
\begin{document}
\section*{题目1}
\begin{enumerate}[label=({\arabic*}), leftmargin=*, itemsep=0.2em, parsep=0pt, topsep=0.4em]
    \item $!\,(\sim x)$
    \item $!\,x$
    \item $!\,((x \& 0xFF)\,\char`^\,0xFF)$
    \item $!\,(\sim((x \& 0xFF000000)\,|\,0xFFFFFF))$
\end{enumerate}

\section*{题目2}
\begin{enumerate}[label=({\alph*}), leftmargin=*, itemsep=0.2em, parsep=0pt, topsep=0.4em]
    \item 恒为1。dx是x的精确表示,因此float(x)与float(dx)均为将x表示为float格式,且舍入方式相同。
    \item 不恒为1。x为INT\_MAX,y=-1时x-y会溢出,结果为INT\_MIN。而dx-dy为INT\_MAX+1。
    \item 恒为1。int转为double可以实现精确表示,而且double的frac位数大,满足结合律。
    \item 不恒为1。double的frac位数为53位,而三个32位整型的乘积可能会造成舍入,因此不满足结合律。
    \item 不恒为1。x=0的时候,dx/dx结果为NaN,该表达式恒为0。
\end{enumerate}

\section*{题目3}
%添加一段代码
\begin{verbatim}
float_bits float_absval (float_bits f) {
    float_bits exp = (f >> 23) & 0xFF;
    float_bits frac = f & 0x7FFFFF;
    if (exp == 0xFF && frac != 0) return f;
    else return (f & 0x7FFFFFFF);
}
\end{verbatim}

\section*{题目4}
\begin{verbatim}
float_bits float_twice(float_bits f) {
    float_bits s = (f >> 31) << 31;
    float_bits exp = (f >> 23) & 0xFF;
    float_bits frac = f & 0x7FFFFF;
    if (exp == 0xFF) return f;
    if (exp != 0){ // 规格化数
        exp = exp + 1;
        if (exp == 0xFF) // 溢出为无穷
            frac = 0;
    }
    else { // 非规格化数
        frac = frac << 1;
        if (frac & 100000){
            exp = 0;
            frac = frac & 0x7FFFFF;
        }
    }
    return (s | (exp << 23) | frac);
}
\end{verbatim}
\end{document}