\documentclass[UTF8]{ctexart}
\usepackage{geometry}
\usepackage{titling}
\usepackage{listings}
\usepackage{xcolor}
\usepackage{graphicx}
\usepackage{float}
\usepackage{hyperref}

% Font setup: use XeLaTeX (fontspec + xeCJK) to provide proper CJK bold fonts
% Compile with XeLaTeX (e.g. `xelatex 实验报告.tex`) to make these effective.
\usepackage{fontspec}
\usepackage{xeCJK}
% Set CJK main font and ensure a bold font is available (SimSun / SimHei on Windows)
\setCJKmainfont[BoldFont={SimHei}]{SimSun}
\setCJKsansfont{SimHei}
% Monospaced font for code listings
\setmonofont{Consolas}

\geometry{a4paper,scale=0.8}
\setlength{\droptitle}{-3cm}
\title{最短路径中文文本分词 实验报告}
\author{姓名:王松宸 \quad 学号:2024201594}
\date{\today}

% Ensure no headers, only footer page numbers (bottom center)
\pagestyle{plain}

\begin{document}

\maketitle

\section{系统概述与功能亮点}

\subsection{简介}
我设计并实现了一个高性能的中文分词系统,该系统基于大规模词典和有向无环图,采用 N-最短路径算法对中文句子进行切分。系统不仅能够输出最优的分词结果,还能根据用户需求输出概率最高的 N 条分词路径。

\subsection{功能亮点}
程序在完成所有基本功能的同时,\textbf{完整实现了实验规定的拓展内容},亮点如下:

\begin{itemize}
    \item \textbf{高效词典管理}:
    \begin{itemize}
        \item 实现了基于链地址法的哈希表,采用 BKDR Hash 算法,支持动态扩容,能够高效加载和查询数十万条词条的大规模词典。
        \item 支持词频和词性信息的存储,后续用于权重调整。
    \end{itemize}
    \item \textbf{智能分词算法}:
    \begin{itemize}
        \item \textbf{DAG 构建}:根据输入句子构建词语切分有向无环图,边权结合了词频概率与词性权重。
        \item \textbf{N-最短路径}:实现了基于动态规划(DP)的 N-Best 算法,能够计算并回溯出前 N 条最优路径,而非仅仅是单条最短路径。
        \item \textbf{词性加权}:设计了精细的词性权重调整策略,针对成语、专有名词、副词等进行降权(优先匹配),对虚词进行惩罚,有效解决了歧义切分问题。
    \end{itemize}
    \item \textbf{美观的交互界面}:
    \begin{itemize}
        \item 设计了基于 CLI 的图形化菜单,使用 ASCII 字符构建边框,提升了系统的专业感。
        \item 引入 ANSI 颜色控制,对不同类型的输出(如菜单、提示符、错误信息、分词结果)进行颜色区分,极大地增强了可读性和用户体验。
    \end{itemize}
\end{itemize}

\subsection{输入与输出内容}
\begin{itemize}
    \item \textbf{输入}:用户交互式输入的中文句子及 N 值(路径数量)。
    \item \textbf{输出}:按概率从高到低排序的 N 条分词结果,词与词之间用“/”分隔。
\end{itemize}

\section{程序架构}

\subsection{词典模块}
该模块负责管理核心词库数据,主要包含初始化和查询两个阶段:
\begin{itemize}
    \item \textbf{初始化}:读取 \texttt{dict\_big.txt},解析每行的词语、词频和词性,将其插入哈希表中。
    \item \textbf{查询}:提供高效的字符串匹配接口,用于在图构建过程中快速判断子串是否成词。
\end{itemize}

\subsection{图构建模块}
该模块负责将线性的句子转化为结构化的图模型:
\begin{itemize}
    \item \textbf{节点映射}:将句子中的每个字符位置映射为图的一个节点。
    \item \textbf{边生成}:遍历句子的所有子串,若子串在词典中存在,则在对应起止位置之间建立一条有向边。
    \item \textbf{权重计算}:边的权重由词频概率($-\ln P$)和词性系数共同决定,实现了语义感知的路径规划。
\end{itemize}

\subsection{算法核心模块}
该模块实现了 N-最短路径的核心逻辑:
\begin{itemize}
    \item \textbf{拓扑排序与 DP}:利用 DAG 的拓扑特性,按顺序遍历节点,计算到达每个节点的前 N 个最短路径代价。
    \item \textbf{路径回溯}:根据记录的前驱节点信息,从终点反向回溯出完整的切分路径。
\end{itemize}

\subsection{前端交互模块}
该模块负责与用户进行交互,提供友好的操作界面:
\begin{itemize}
    \item \textbf{界面渲染}:利用 ANSI 转义序列实现控制台文本的颜色高亮和样式控制(如加粗),绘制美观的菜单边框。
    \item \textbf{交互逻辑}:封装了清晰的主循环逻辑,支持菜单选择、输入验证、清屏刷新等功能,确保用户操作流畅。
    \item \textbf{结果展示}:将分词结果以高亮形式输出,清晰区分路径编号与分词内容。
\end{itemize}

\section{核心设计}

\subsection{数据结构定义}
\begin{lstlisting}[language=C++]
// 哈希表节点
typedef struct DictNode {
    char *word;            // 词
    int frequency;         // 词频
    char *pos;             // 词性
    struct DictNode *next; // 链表指针
} DictNode;

// 图的边结构
struct Edge {
    int to;            // 指向的节点索引
    double weight;     // 权重
    char word[128];    // 词的内容
    char pos[64];      // 词性
    struct Edge *next; // 邻接表指针
};
\end{lstlisting}

\subsection{关键算法实现}

\subsubsection{基于 BKDR Hash 的词典索引}
为了在构建图时快速查找词语,在AI的建议下采用了经典的 BKDR Hash 算法:
\begin{itemize}
    \item 选用 seed 为 131,能够极大地减少哈希冲突。
    \item 仿照教材采用哈希容量递增表,实现了自动扩容机制,当负载因子超过 0.95 时自动扩大哈希表容量并重哈希,保证查找效率维持在 $O(1)$ 水平。
\end{itemize}

\subsubsection{N-最短路径算法}
本系统利用了分词图的有向无环(DAG)特性,采用动态规划实现:
\begin{itemize}
    \item \textbf{状态定义}:$D[i][k]$ 表示到达第 $i$ 个字符位置的第 $k$ 短路径的长度。
    \item \textbf{状态转移}:对于节点 $i$ 的每一条出边 $(i, to)$,尝试将其扩展到 $D[to]$ 的候选路径集中。
    \item \textbf{有序插入}:在更新 $D[to]$ 时,采用插入排序的思想,维护一个大小为 N 的有序数组,始终保留当前最优的 N 个解。
\end{itemize}

\subsubsection{基于词性的权重系数}
为了解决中文分词中的歧义问题(如“的确”被切分为“的/确”),设计了 \texttt{GetPosMultiplier} 函数:
\begin{itemize}
    \item \textbf{优先匹配}:对习用语(i)、成语(l)、专有名词(nr/ns/nt)赋予极低的权重系数,使其在路径竞争中占据优势。
    \item \textbf{歧义修正}:将副词(d)的权重系数设为 0.6,鼓励“的确”等副词作为一个整体被切分。
    \item \textbf{虚词惩罚}:对助词(u)、介词(p)等高频虚词赋予 1.2 的权重系数,防止长词被过度切碎。
\end{itemize}

\section{调试分析}

\subsection{歧义切分问题的解决}
\begin{itemize}
    \item \textbf{问题}:在测试句子“这的确是一个好主意”时,系统倾向于将“的确”切分为“的/确”,因为“的”字频率极高,导致其路径权重较小。
    \item \textbf{解决}:引入了词性加权机制。通过查阅词典发现“的确”的词性为副词(d),因此在代码中特判了副词的权重(0.6),同时对单字虚词(u)施加惩罚(1.2)。调整后,系统成功输出了“这/的确/是/...”的正确结果。同时也借此对其他词性进行了合理的权重分配,提升了整体分词准确率。
\end{itemize}

\subsection{malloc 与 new 的选择}
\begin{itemize}
    \item \textbf{问题}:在创建各种数据结构时,我更倾向于选择更熟练使用的 \texttt{malloc} 进行内存分配。然而,这也让我在释放内存部分中遇到了各种问题,导致好多次的异常报错。
    \item \textbf{解决}:
    \begin{itemize}
        \item 二维及以上的动态数组,采用多重循环逐层释放,确保每一层的内存都被正确释放。
        \item 下次最好使用 \texttt{new} 和 \texttt{delete},以避免手动管理内存时的复杂性和潜在错误。
    \end{itemize}
\end{itemize}

\section{功能测试}
\subsection{测试结论}
经过对多种类型的中文句子(包括日常用语、成语、长难句)进行测试,得出结论:
\begin{itemize}
    \item 系统启动迅速,词典加载稳定。
    \item 分词准确率高,能够正确处理常见的歧义组合。
    \item N-最短路径功能正常,能够输出合理的备选分词方案。
\end{itemize}

\subsection{功能测试报告}
详细测试报告请参阅 \textbf{\texttt{功能测试报告.pdf}}。

\subsection{使用手册}
详细的操作指南请参阅 \textbf{\texttt{使用手册.pdf}}。

\section{附录}
\begin{itemize}
    \item \texttt{lab4.cpp}
    \item 功能测试报告.pdf
    \item 使用手册.pdf
\end{itemize}
\end{document}
