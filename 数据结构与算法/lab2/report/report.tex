\documentclass[12pt,a4paper]{article}
\usepackage[UTF8]{ctex}
\usepackage{geometry}
\usepackage{hyperref}
\usepackage{enumitem}
\usepackage{titling}
\usepackage{titlesec}
\usepackage{listings}
\usepackage{xcolor}
\geometry{margin=1in}
\hypersetup{colorlinks=true,linkcolor=blue,urlcolor=blue}

% 调整标题垂直位置(负值上移,按需修改)
\setlength{\droptitle}{-3cm}

% 使所有 section标题居中
% \titleformat{\section}[block]{\centering\Large\bfseries}{\thesection}{1em}{}

\title{HTML 结构解析与校验实验报告}
\author{王松宸 2024201594}
\date{\today}

\lstdefinestyle{code}{
	basicstyle=\ttfamily\small,
	breaklines=true,
	frame=single,
	backgroundcolor=\color[RGB]{248,248,248},
	keywordstyle=\color{blue},
	commentstyle=\color{gray},
	numbers=left,
	numberstyle=\tiny\color{gray},
	tabsize=2,
	showstringspaces=false
}

\begin{document}
\maketitle

\section{核心内容}
本实验基于线性表与栈,完成对 HTML 文档的读取、合法性校验(CheckHTML),并实现了OuterHTML与Text两个查询功能。

此外,\textbf{扩展实现了额外功能(a),支持输入绝对路径、相对路径与部分路径。}

同时,\textbf{针对用户实现了大小写不敏感匹配等功能},提升了查询的灵活性与实用性。

\section{代码总体设计}
程序采用“单遍扫描 + 栈”策略:
\begin{enumerate}[nosep]
	\item 读取 HTML 文件(去除 UTF-8 BOM),缓存为 \verb|g_html|;
	\item 预处理阶段进行基础配对校验:遇到起始标签入栈,遇到结束标签出栈;完整跳过注释以及 \verb|<script>|/\verb|<style>| 块;将校验错误记录在 \verb|g_errors| 中;
	\item 交互命令:
	\begin{itemize}[nosep]
		\item CheckHTML:输出配对、嵌套、未闭合等错误;
		\item Outer\_HTML(path):按简化 XPath 匹配,输出对应元素的 OuterHTML;
		\item Text(path):在匹配元素内抽取文本,按块级元素插入换行并合并多余空白。
	\end{itemize}
\end{enumerate}

核心模块关系如下:文件读取 \(\to\) 预处理校验 \(\to\) 路径解析与匹配 \(\to\) OuterHTML/Text 输出。

\section{数据结构与关键函数解析}
\subsection{栈结构}
顺序栈 \verb|SqStack|:
\begin{itemize}[nosep]
	\item 成员:\verb|base|、\verb|top|、\verb|stacksize|;
	\item \verb|initStack| 初始化容量为 \verb|MAXSIZE|;\verb|Push| 满则按 \verb|MAXSIZE| 增量扩容;\verb|Pop| 空栈返回错误;
	\item 额外提供 \verb|PushTag|/\verb|PopTag| 封装标签名的入栈/出栈(以 \verb|'\0'| 作为分隔)。
\end{itemize}

\subsection{路径解析与匹配(实现额外功能)}
\begin{itemize}[nosep]
	\item \textbf{parse\_path\_tokens}:按 \verb|'/'| 切分,去除多余空白;识别三类情形:
	\begin{enumerate}[nosep]
		\item \textit{全部选择}:路径为空或为 \verb|'/'|,表示选择整个文档;
		\item \textit{绝对路径}:以 \verb|'/'| 开头,要求路径与当前打开路径\textbf{完全相等};
		\item \textit{相对/部分路径}:不以 \verb|'/'| 开头,允许路径作为当前打开路径的\textbf{后缀}匹配(即部分路径)。
	\end{enumerate}
	\item \textbf{path\_matches}:实现上述两种匹配策略;
	\item \textbf{perform\_query}:扫描文档维护打开标签栈(名称序列),在遇到起始/结束标签时检查是否与目标路径匹配:
	\begin{itemize}[nosep]
		\item Outer\_HTML:在匹配的起始标签处,找到与之配对的结束位置,输出完整片段;
		\item Text:在匹配范围内调用文本抽取函数,输出归一化文本。
	\end{itemize}
\end{itemize}

\subsection{标签解析}
\begin{itemize}[nosep]
	\item \textbf{find\_tag\_gt}:从字符 \verb|'<'| 起,正确跨越属性单引号/双引号,找到配对的 \verb|'>'|;
	\item \textbf{skip\_comment}:跳过 \verb|<!-- ... -->| 整段;
	\item \textbf{skip\_script\_style\_block}:匹配对应的关闭标签,整体跳过脚本/样式块(不参与栈与内容抽取);
	\item \textbf{自闭合识别}:内置常见自闭合标签集合(BR/HR/IMG/META/LINK/INPUT/AREA/...),并支持显式形式 \verb|<tag ... />|;
	\item \textbf{大小写处理}:统一将标签名提升为大写参与比较,匹配不区分大小写。
\end{itemize}



\subsection{文本抽取与归一化}
\begin{itemize}[nosep]
	\item \textbf{extract\_text\_in\_range}:忽略标签/注释/脚本样式,累积文本;遇到块级元素(HTML/BODY/DIV/P/UL/LI/TABLE/TD/TH/\dots)闭合时插入换行;\verb|<br>| 直接视为换行;
	\item \textbf{normalize\_text\_preserve\_newlines}:
	\begin{itemize}[nosep]
		\item 连续空格、\verb|\t|、\verb|\r| 折叠为单个空格;
		\item 保留并折叠我们插入的换行符,逐行修剪首尾空格;
		\item 移除末尾多余换行,确保输出整洁稳定。
	\end{itemize}
\end{itemize}

\subsection{预处理与内容校验}
	extbf{preprocess\_html} 负责:
\begin{itemize}[nosep]
	\item 线性扫描输入,遇起始标签入栈、结束标签出栈,记录未闭合/多余关闭等结构性错误;
	\item 对注释与 \verb|<script>|/\verb|<style>| 采取整体跳过,避免误报配对;
	\item 以 \verb|OpenTag| 向量辅助报错位置输出。
\end{itemize}
	extbf{validate\_content\_model} 对内容模型做检查,示例规则:
\begin{itemize}[nosep]
	\item R1:若当前为块级元素,祖先若为行内且不为 \verb|A|,提示可能的嵌套不当;
	\item R2:祖先为 \verb|H1–H6|/\verb|P|/\verb|DT| 时,不允许再出现块级后代;
	\item R3:\verb|A| 不能包含 \verb|A| 后代。
\end{itemize}

\subsection{文件读入与 BOM 处理}
	extbf{load\_file} 以二进制读取本地文件,统一移除 UTF-8 BOM,将内容缓存到全局 \verb|g_html| 并以 \verb|'\0'| 结尾,便于后续以 \verb|string_view| 只读视图进行高效扫描。

\section{命令接口说明}
\begin{description}
	\item[1 解析新的文件] 读取本地 HTML 文件到内存,自动去除 UTF-8 BOM。
	\item[2 检查此文件合法性] 利用栈完成配对、嵌套与未闭合检查,列出错误位置与标签名。
	\item[3 输出对应路径下的 html 代码段] \verb|Outer_html(path)|:多个匹配节点用换行分隔。
	\item[4 输出对应路径下的文本] \verb|Text(path)|:按块级元素换行策略输出整洁文本。
	\item[5 退出程序] 结束交互。
\end{description}

\section{面向用户的设计与交互细节}
\begin{description}
	\item \textbf{1 宽容的路径输入}:路径前后空白会被修剪,多余的 \verb|'/'| 自动忽略;标签名大小写不敏感;空路径或 \verb|'/'| 表示“整页”。
	\item \textbf{2 清晰的提示与回显}:未加载文件时阻止查询并提示操作顺序;每次命令后重新打印菜单,降低误操作成本。
	\item \textbf{3 输出一致性}:多个匹配结果以单个换行分隔;文本抽取遵循“块级换行、空白合并、行级修剪”的格式化规则,便于直读与比对。
	\item \textbf{4 错误处理}:
		\begin{itemize}[nosep]
			\item 文件不存在/无法打开时及时报错;
			\item 结构性错误(未闭合、错位闭合、脚本/样式缺关闭等)集中展示;
			\item 查询无命中时输出提示而非静默失败。
		\end{itemize}
\end{description}

\section{总结}
本实验以顺序栈为核心,完成了 HTML 的结构校验与两类查询;
在此基础上实现了XPath额外功能(绝对/相对/部分路径)与用户友好功能。

\end{document}
