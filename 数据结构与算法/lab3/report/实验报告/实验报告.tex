\documentclass[UTF8]{ctexart}
\usepackage{geometry}
\usepackage{titling}
\usepackage{listings}
\usepackage{xcolor}
\usepackage{graphicx}
\usepackage{float}
\usepackage{hyperref}

% Font setup: use XeLaTeX (fontspec + xeCJK) to provide proper CJK bold fonts
% Compile with XeLaTeX (e.g. `xelatex 实验报告.tex`) to make these effective.
\usepackage{fontspec}
\usepackage{xeCJK}
% Set CJK main font and ensure a bold font is available (SimSun / SimHei on Windows)
\setCJKmainfont[BoldFont={SimHei}]{SimSun}
\setCJKsansfont{SimHei}
% Monospaced font for code listings
\setmonofont{Consolas}

\geometry{a4paper,scale=0.8}
\setlength{\droptitle}{-3cm}
\title{HTML CSS Selector 实验报告}
\author{姓名:王松宸 \quad 学号:2024201594}
\date{\today}

% Ensure no headers, only footer page numbers (bottom center)
\pagestyle{plain}

\begin{document}

\maketitle

\section{系统概述与功能亮点}

\subsection{简介}
我设计并实现了高性能、功能完备的HTML 解析器和 CSS 选择器,能够将非结构化的 HTML 文本解析为结构化的树,并提供 CSS 选择器和 XPath 两种强大的查询接口,支持对网页内容的精准定位、提取与修改。

\subsection{功能亮点}
程序在完成所有基本功能的同时,\textbf{完整实现了实验规定的所有选做内容},功能特性如下:

\begin{itemize}
    \item \textbf{HTML 解析}:支持标准 HTML 标签、属性解析,能够正确处理自闭合标签及文本结点,构建完整的树。
    \item \textbf{双路径查询}:
    \begin{enumerate}
        \item \textbf{CSS 选择器}:
        \begin{itemize}
            \item \textbf{基础支持}:标签、ID、类、通配符选择器。
            \item \textbf{组合支持}:后代(空格)、分组(逗号)、相邻兄弟(\texttt{+})、通用兄弟(\texttt{\textasciitilde})、子元素(\texttt{>})。
            \item \textbf{高级伪类与伪元素}:实现了 \texttt{:empty}、\texttt{:first-child}、\texttt{::first-letter}、\texttt{::first-line}、\texttt{::before}、\texttt{::after}六个选做的功能。
        \end{itemize}
        \item \textbf{XPath 查询}:
        \begin{itemize}
            \item 实现了 XPath 核心语法,支持绝对路径与相对路径。
            \item 支持属性谓语过滤(如 \texttt{[@class='news']})及索引定位(如 \texttt{//li[1]})。
            \item 支持 \texttt{text()} 函数直接提取文本结点。
        \end{itemize}
    \end{enumerate}
\end{itemize}

\subsection{输入与输出内容}
\begin{itemize}
    \item \textbf{输入}:HTML 源文件路径及用户交互式的查询指令。
    \item \textbf{输出}:结果结点列表、InnerText 文本提取结果、OuterHTML 源码结果、A 标签所有链接。
\end{itemize}

\section{程序架构}

\subsection{解析模块}
该模块负责将非结构化的 HTML 源码转换为一棵完整的树,主要包含预处理和解析两个阶段:
\begin{itemize}
    \item \textbf{预处理}:读取 HTML 文件,去除无关字符,处理编码问题。
    \item \textbf{解析}:采用基于栈的单遍扫描算法,能够正确处理标签的嵌套关系、自闭合标签(如 \texttt{<img>})以及文本,将各部分正确地存储在结点内。同时具备一定的容错能力,能够忽略注释与 DOCTYPE 声明。
\end{itemize}

\subsection{查询模块}
该模块实现了基础功能和选做功能,包含两个独立的查询引擎:
\begin{itemize}
    \item \textbf{CSS 选择器引擎}:支持从基础选择器到复杂组合器(后代、兄弟、子元素)的解析与执行,并集成了伪类/伪元素的高级逻辑。
    \item \textbf{XPath 引擎}:支持路径导航(\texttt{/} 与 \texttt{//})、属性谓语过滤以及索引定位。
\end{itemize}

\subsection{结果处理模块}
该模块负责对查询模块返回的内部结点列表进行进一步的处理。
\begin{itemize}
    \item \textbf{再查询}:支持对结果列表里的特定结点进行进一步的 CSS / XPath 查询。
    \item \textbf{文本提取}:实现了 \texttt{innerText},能够提取特定结点下的所有文本内容。
    \item \textbf{源码提取}:实现了 \texttt{outerHTML},能够输出特定结点的完整 HTML 源码。
    \item \textbf{链接提取}:针对 \texttt{<a>} 标签,能够批量提取所有链接地址。
\end{itemize}

\section{核心设计}

\subsection{数据结构定义}
\begin{lstlisting}[language=C++]
struct Node
{
    NodeType type;             // 结点类型
    char *tag_name;            // 标签名:div, p, a等 (仅元素结点)
    char *id;                  // id属性
    char **classes;            // class列表 (字符串数组)
    int class_count;           // class数量
    char *href;                // href属性 (仅a标签)
    char *text;                // 文本内容 (仅文本结点)
    struct Node *parent;       // 父结点指针
    struct Node *children;     // 第一个子结点指针 (左孩子)
    struct Node *next_sibling; // 下一个兄弟结点指针 (右兄弟)
};
\end{lstlisting}

\subsection{关键算法实现}

\subsubsection{基于栈的 HTML 解析算法}
采用一次遍历策略,利用栈结构维护标签嵌套关系:
\begin{itemize}
    \item 遇到开始标签入栈,遇到结束标签出栈。
    \item 遇到自闭合标签(如 \texttt{img}, \texttt{meta})直接挂载到当前栈顶元素下,不入栈。
    \item 解析器能够自动忽略 HTML 中的注释、DOCTYPE 声明,并保留必要的空白文本结点以维持排版格式。
\end{itemize}

\subsubsection{CSS 选择器设计}
CSS 查询引擎采用“分词-过滤”的结构:
\begin{itemize}
    \item \textbf{解析阶段}:将复杂选择器(如 \texttt{div\#main > p.active})分解为原子选择器序列和组合符序列。
    \item \textbf{执行阶段}:维护一个“当前候选结点集合”。每处理一个组合符(如 \texttt{>}),就基于当前集合中的结点,查找其符合条件的子结点,生成新的候选集合。
    \item \textbf{高级特性实现}:
    \begin{itemize}
        \item \textbf{兄弟选择器 (\texttt{+} / \texttt{\textasciitilde})}:在遍历 \texttt{next\_sibling} 链表时,增加了智能跳过 \texttt{TEXT\_NODE}(空白换行符)的逻辑,确保准确匹配下一个元素结点。
        \item \textbf{伪元素 (\texttt{::first-line})}:实现了文本自动截断算法,能够智能识别并提取首行文本,忽略前导空白。
    \end{itemize}
\end{itemize}

\subsubsection{XPath 递归下降搜索}
XPath 引擎基于递归思想实现:
\begin{itemize}
    \item \textbf{路径解析}:将完整路径解析为路径段和谓语条件,比如对于 \texttt{//div[@id='a']},路径段为 \texttt{div},谓语条件为 \texttt{id='a'}。
    \item \textbf{递归搜索}:
    \begin{itemize}
        \item 若路径以 \texttt{/} 开头,仅在当前上下文的直接子结点中匹配。
        \item 若路径以 \texttt{//} 开头,则遍历当前上下文的整棵子树(深度优先)。
    \end{itemize}
    \item \textbf{索引支持}:在匹配过程中维护计数器,支持 \texttt{[n]} 语法精确定位同级元素中的第 n 个。
\end{itemize}

\section{调试分析}

\subsection{HTML 源码中空白结点的干扰}
\begin{itemize}
    \item \textbf{问题}:HTML 源码格式化产生的换行符和缩进会被解析为 \texttt{TEXT\_NODE}。这导致 \texttt{node->next\_sibling} 往往是空白文本而非下一个标签,直接破坏了 \texttt{+} 选择器和 \texttt{:first-child} 的判定逻辑。
    \item \textbf{解决}:在底层遍历逻辑中封装了 \texttt{get\_next\_element\_sibling()} 和 \texttt{get\_first\_element\_child()} 辅助函数,显式跳过非 \texttt{ELEMENT\_NODE} 类型的结点。
\end{itemize}

\subsection{伪元素和伪类相关功能的实现}
\begin{itemize}
    \item \textbf{问题}:扩展的六个功能完全无法套用原先的处理逻辑
    \item \textbf{解决}:解析命令时补充了对伪类和伪元素的操作,并在 \texttt{match\_part} 和 \texttt{traverse\_and\_match}函数中都进行了特判处理,保证所有功能都得到具体实现。
\end{itemize}

\section{功能测试}
\subsection{测试结论}
经过对标准网页及复杂新闻页面的全面测试,我们可以得出结论:
\begin{itemize}
    \item 程序解析功能稳定,无内存泄漏。
    \item CSS 选择器支持所有基础及选做的高级语法。
    \item XPath 选择器能够准确执行路径查找和属性过滤。
\end{itemize}

\subsection{功能测试报告}
详细测试报告请参阅 \textbf{\texttt{功能测试报告.pdf}}。

\subsection{使用手册}
详细的操作指南请参阅 \textbf{\texttt{使用手册.pdf}}。

\section{附录}
\begin{itemize}
    \item \texttt{lab3.cpp}
    \item 功能测试报告.pdf
    \item 使用手册.pdf
\end{itemize}
\end{document}
