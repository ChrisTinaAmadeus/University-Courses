\documentclass{article}
\usepackage[utf8]{inputenc}
\usepackage{ctex} % 支持中文
\usepackage{graphicx}
\usepackage{listings}
\usepackage{xcolor}
\usepackage{hyperref}
\usepackage{geometry}
\usepackage{titling}

\geometry{a4paper, scale=0.8}

\title{语音合成实验报告}
\author{王松宸 2024201594}
\date{\today}
\setlength{\droptitle}{-3cm} % 调整标题位置
\begin{document}

\maketitle

\section{实验目的}
本次实验旨在使用现成的 FastSpeech 2 模型,理解非自回归语音合成模型的原理。通过配置环境、下载预训练模型,最终实现英文和中文的文本到语音(TTS)合成。

\section{模型理解}

在阅读代码后,我对 FastSpeech 2 有了以下几点直观的理解:

\subsection{为什么叫 "Fast"?}
传统的 TTS 模型(如 Tacotron 2)是“自回归”的,也就是说,它生成第 2 个时刻的声音时,必须依赖第 1 个时刻的结果。

而 FastSpeech 2 是“非自回归”的(Parallel),它像是一个统筹全局的指挥官,一次性就能预测出整句话的所有频谱帧。这使得它的推理速度极快,非常适合实时应用。

\subsection{核心组件:Variance Adaptor}
查阅资料后我发现,FastSpeech 2 相比一代最大的改进在于引入了 \textbf{Variance Adaptor(变量适配器)}。它不再只是简单地把字变成声音,而是试图去预测声音的“抑扬顿挫”。模型中包含了三个关键的预测器:
\begin{itemize}
    \item \textbf{Duration Predictor(时长预测):} 决定每个音素(比如一个拼音)读多长时间。
    \item \textbf{Pitch Predictor(音高预测):} 决定声音的高低起伏。
    \item \textbf{Energy Predictor(能量预测):} 决定声音的响度。
\end{itemize}

正是这些组件,让合成出来的声音不再是冷冰冰的机器音,而是有了情感和节奏。

\section{实验过程}

本次实验最大的挑战不在于运行代码,而在于配置好所有环境。

\subsection{环境搭建}
由于项目依赖较旧(PyTorch 1.7),直接在 Base 环境安装会导致各种冲突。我采取了以下策略:
\begin{itemize}
    \item \textbf{使用 Miniconda 管理环境:} 在 WSL 中安装 Miniconda,创建了一个基于 Python 3.8 的独立环境 \texttt{fs2}。
    \item \textbf{解决依赖冲突:} 在安装 \texttt{pyworld} 和 \texttt{numpy} 时遇到了严重的编译错误。最终发现必须使用 Conda 安装二进制包来规避源码编译的兼容性问题。
    \item \textbf{解决库冲突(Double Free):} 在合成时遇到了 Linux 下经典的 \texttt{free(): double free detected} 错误。我最终通过卸载 pip 版 torch,转而安装官方 CPU 版 torch 解决了这个问题。
\end{itemize}

\subsection{模型准备}
\begin{itemize}
    \item \textbf{声码器:} 解压了项目自带的 HiFi-GAN 模型(\texttt{generator\_LJSpeech} 和 \texttt{generator\_universal})。
    \item \textbf{声学模型:} 下载了 LJSpeech(英文)和 AISHELL3(中文)的预训练 Checkpoint,并按要求重命名为 \texttt{restore\_step.pth.tar} 格式,放置在 \texttt{output/ckpt/} 目录下。
\end{itemize}

\subsection{代码调试}
\begin{itemize}
\item\textbf{修复音频截断问题:}
初次合成时,发现音频的首尾部分有明显的吞字现象。我在在音素序列(Phoneme Sequence)的前后添加了 \texttt{\{sp\}}(静音标记),成功解决了截断问题。

\item \textbf{修复文件名过长导致无法创建文件的问题:}
当输入长文本时,Linux 无法创建过长的文件名。我修改了保存逻辑,自动截取文件名的前 15 个字符。
\end{itemize}

\section{实验结果}
实验成功生成了多条清晰的 \texttt{.wav} 音频文件。
\begin{itemize}
    \item \textbf{英文效果:} LJSpeech 模型生成的英文发音标准,连读自然,HiFi-GAN 声码器还原的音质非常高,几乎听不出底噪。
    \item \textbf{中文效果:} AISHELL3 模型支持多说话人。通过调整 \texttt{speaker\_id},我可以生成不同音色的声音。虽然部分长句的韵律感稍显机械,但字正腔圆,清晰度极佳。
\end{itemize}

\section{个人心得}
这次实验让我深刻体会到了“配置环境是深度学习的第一道门槛”。

1. \textbf{环境隔离:} 为了避免破坏我的base环境,我专门为此学习了 Miniconda 的使用。通过创建独立环境,可以自由安装各种版本的库,而不担心冲突。

2. \textbf{对 End-to-End 的思考:} FastSpeech 2 虽然叫 End-to-End,但其实还是分成了“声学模型”和“声码器”两步。声学模型预测 Mel 频谱,声码器把频谱变回波形。这种解耦的设计让模型训练更稳定,也方便替换不同的声码器。

3. \textbf{实践出真知:} 只有亲自跑代码,才会遇到各种各样的问题。解决问题的过程正是提高我科研能力的过程。

\end{document}