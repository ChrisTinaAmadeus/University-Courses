\documentclass[12pt,a4paper]{article}
\usepackage[margin=1in]{geometry}
\usepackage{xeCJK}
\setCJKmainfont{SimSun}
\usepackage{fontspec}
\setmainfont{Times New Roman}
\usepackage{amsmath, amssymb}
\usepackage{graphicx}
\usepackage{xcolor}
\usepackage{hyperref}
\usepackage{float}
\usepackage{listings}
\usepackage{caption}

% 代码高亮样式
\lstdefinestyle{mypython}{
	language=Python,
	basicstyle=\ttfamily\small,
	keywordstyle=\color{blue!70!black},
	stringstyle=\color{green!50!black},
	commentstyle=\color{gray},
	showstringspaces=false,
	numbers=left,
	numberstyle=\tiny\color{gray},
	stepnumber=1,
	numbersep=8pt,
	frame=single,
	framerule=0.5pt,
	breaklines=true,
	tabsize=4
}

% CJK 字体补充:设置等宽与无衬线,减少缺字警告
\setCJKmonofont{SimSun}
\setCJKsansfont{SimSun}
% 等宽英文字体(Windows 常见字体)
\setmonofont{Consolas}

\title{蒙特卡洛法估算圆周率实验报告}
\author{王松宸\\学号:2024201594}
\date{}

\begin{document}
\maketitle

\section{实现步骤}
\begin{enumerate}
	\item 使用 NumPy 生成长度为 $n$ 的随机向量 $x,y\sim\mathcal{U}(0,1)$;
	\item 判断 is\_in\_circle 的值,从而统计圆内数量 $k$;
	\item 计算 $\hat{\pi}=4k/n$ 并输出;
	\item 变更 $n$ 取不同数量,比较估计值的收敛情况;
\end{enumerate}

\section{源代码}

\lstinputlisting[style=mypython]{源码.py}

\section{运行结果}

% 若同目录下不存在 result.png,使用占位框以保证可编译
\IfFileExists{result.png}{
	\begin{figure}[H]
		\centering
		\includegraphics[width=0.9\linewidth]{result.png}
		\caption{不同 $n$ 下的 $\pi$ 估计输出。}
		\label{fig:result}
	\end{figure}
}

\section{结论}
实验表明,随着采样点数 $n$ 的增加,基于蒙特卡洛方法的 $\pi$ 估计值逐步收敛到真实值 $3.1415926\dots$。然而,收敛速度与方差下降较慢,若需要高精度需显著增大样本规模或采用方差缩减技术。

\end{document}