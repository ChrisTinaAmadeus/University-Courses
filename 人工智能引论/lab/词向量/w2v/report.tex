\documentclass[12pt]{article}
\usepackage{ctex}
\usepackage{amsmath, amssymb}
\usepackage{graphicx}
\usepackage{booktabs}
\usepackage{geometry}
\usepackage{hyperref}
% 建议与 hyperref 搭配使用,避免“Rerun to get outlines right”警告
\usepackage{bookmark}
\usepackage{titling}
\geometry{a4paper,margin=2.5cm}
\title{Skip-gram 词向量实验报告}
\author{王松宸 2024201594 }
\date{\today}
\setlength{\droptitle}{-3cm}
% 统一正文段首缩进宽度
\setlength{\parindent}{2em}

\begin{document}
\maketitle

\section{引言}
词向量是将离散的词映射到连续向量空间的一种表示学习方法,相比于 one-hot 表示,词向量在维度上大幅压缩,并通过训练捕获词间的语义与句法关系,使得向量空间中距离与方向具有一定的语义意义,例如类比关系 \(\text{king} - \text{man} + \text{woman} \approx \text{queen}\)。

\section{对词向量的理解}

\subsection{概述}
把每个词放到同一个连续空间里,像在一张看不见的“语义地图”上给它一个坐标。意思越相近、在相似语境里常一起出现的词,坐标就越靠近;用法差很多的词,距离就更远。最终得到的是一张“词到点”的映射表,人们把这些点的坐标称为词向量。

\subsection{词向量与 One-hot 的区别}
One-hot 只会告诉你“词是不是同一个”,不同词之间全都等距,无法表达相似性;而词向量会让“相似的词更近、不相似的更远”,从而能做相似度、聚类、检索等语义相关的事情。另一方面,词向量把信息分布到少量维度上,参数可以共享,数据更容易泛化到没见过的句子里。

\subsection{词向量能做什么}
- 最近邻:找与某词最接近的词,常能得到同义或近义词;

- 类比:在向量空间里,类似于“国王 : 王后”与“男人 : 女人”的对应方向关系,经常能被保留下来;

- 作为下游任务特征:在情感分析、文本分类、序列标注等任务里,词向量可以作为良好的初始化或固定特征。

\section{Skip-gram 模型}

\subsection{概述}
Skip-gram 的目标是:给定一个“中心词”,让模型去预测它周围会一起出现的词。凡是经常出现在同一语境里的词,其向量会被推近;用法差异大的词会被推远。模型内部维护两套嵌入表:输入嵌入代表中心词,输出嵌入代表被预测的上下文词;训练结束后,通常把输入嵌入当作最终的词向量。

\subsection{训练数据如何构造}
遍历语料,按窗口大小在句子上滑动。对于每个中心词,都与窗口内的上下文词组成若干训练样本(中心词, 外部词)。窗口越大,产生的样本越多,语义范围也更广。

\subsection{对学习过程的理解}
对每个样本,模型学习“让正确的上下文词得高分,让无关词得低分”。优化用随机梯度下降完成。

\section{词向量可视化与图像解读}
\label{sec:vis}
\begin{figure}[h]
	\centering
	% 将实际文件名替换为你的词向量投影图片文件,例如 embeddings_pca.png
	\includegraphics[width=0.72\textwidth]{word_vectors.png}
	\caption{word\_vectors}
	\label{fig:embedding_pca}
\end{figure}
\subsection{降维方法}
使用 PCA 将 \(d\)-维嵌入映射到二维。PCA 保留最大方差方向,但轴本身没有固定语义,仅相对位置与方向可供解释。

\subsection{结构观察}
图中 \texttt{king} 与 \texttt{queen} 相对接近且与 \texttt{man}/\texttt{woman} 形成类比方向;积极情感词 (\texttt{amazing}, \texttt{wonderful}, \texttt{great}) 与消极词 (\texttt{bad}, \texttt{annoying}, \texttt{dumb}) 有一定分离。\texttt{male}/\texttt{female} 与 \texttt{man}/\texttt{woman} 邻近显示性别语义子空间。

\subsection{潜在问题}
观察到一些积极词与消极词混杂,可能因训练语料规模有限,导致某些词的上下文信息不足。PCA 投影可能掩盖部分高维结构,可是尝试结合其他降维方法(如 t-SNE)进行补充分析。

\section{梯度结果分析}
\begin{figure}[h]
	\centering
	% 将实际文件名替换为你的损失迭代图片文件,例如 training_loss.png
	\includegraphics[width=0.65\textwidth]{result.png}
	\caption{训练迭代后期的损失值}
	\label{fig:training_loss}
\end{figure}
Loss数值从最初的20左右下降到最终的9.7左右,符合预期的标准,显示出明显的收敛趋势。
从图 2 可见损失在后期围绕一个缓慢下降的趋势波动,说明学习率退火后更新步长较小,进入稳定区域。

\section{结论}
本文以直观视角理解词向量与 Skip-gram,词向量可视化结果中特征较为明显,训练损失稳定收敛。

\end{document}
