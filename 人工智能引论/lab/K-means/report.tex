\documentclass[a4paper]{article}

% --- Packages ---
\usepackage{titling}
\usepackage[a4paper, margin=1in]{geometry}
\usepackage{ctex}
% 字体设置:在使用 pdfLaTeX 编译时,下面的包可以减少或消除 Computer Modern 字体的替换警告。
% 如果你使用 XeLaTeX 或 LuaLaTeX(推荐用于中文),可直接用 XeLaTeX 编译并不需要这些包。
\usepackage[T1]{fontenc}
\usepackage{lmodern}
\usepackage{fix-cm}
\usepackage{graphicx}
\usepackage{amsmath}
\usepackage{float}
\usepackage{caption}
\usepackage{listings}
\usepackage{hyperref}
\usepackage{xcolor}

% --- Document Information ---
\title{K-means聚类实验报告}
\author{王松宸 2024201594}
\date{\today}

\setlength{\droptitle}{-3cm}
% --- Listings Style for Python Code ---
\definecolor{codegreen}{rgb}{0,0.6,0}
\definecolor{codegray}{rgb}{0.5,0.5,0.5}
\definecolor{codepurple}{rgb}{0.58,0,0.82}
\definecolor{backcolour}{rgb}{0.95,0.95,0.92}

\lstdefinestyle{mystyle}{
    backgroundcolor=\color{backcolour},   
    commentstyle=\color{codegreen},
    keywordstyle=\color{magenta},
    numberstyle=\tiny\color{codegray},
    stringstyle=\color{codepurple},
    basicstyle=\ttfamily\footnotesize,
    breakatwhitespace=false,         
    breaklines=true,                 
    captionpos=b,                    
    keepspaces=true,                 
    numbers=left,                    
    numbersep=5pt,                  
    showspaces=false,                
    showstringspaces=false,
    showtabs=false,                  
    tabsize=2
}
\lstset{style=mystyle}

% --- Begin Document ---
\begin{document}

\maketitle

\section{K-means 算法实现}

K-means 算法是一个迭代过程,主要包含以下四个步骤:
\begin{enumerate}
    \item \textbf{初始化}:从数据集中随机选择 $k$ 个样本点作为初始的聚类中心。
    \item \textbf{分配}:对于数据集中的每一个样本点,计算它到 $k$ 个聚类中心的距离(使用欧氏距离的平方),并将其分配给距离最近的聚类中心所在的簇。
    \item \textbf{更新}:对于每一个簇,重新计算其所有样本点的均值,并将该均值作为新的聚类中心。
    \item \textbf{收敛判断}:重复执行分配和更新步骤,直到聚类中心的位置变化小于预设的阈值,或者达到最大迭代次数,算法收敛。
\end{enumerate}

\section{实验设置}
\subsection{数据集与特征}
实验数据来源于猫/狗图像数据集。我使用 \texttt{load\_data.py} 保存图像数据的路径和标签,并用 \texttt{wenlan\_feature\_extractor.py} 提取图像的特征向量并保存。特征文件包含三部分:
\begin{itemize}
    \item \textbf{features}: $N \times D$ 的特征矩阵,其中 $N$ 是样本数,$D$ 是特征维度。
    \item \textbf{labels}: 样本的真实标签(猫0/狗1)。
    \item \textbf{paths}: 每个特征对应的原始图片路径。
\end{itemize}

\subsection{数据预处理}
在聚类前,我对特征进行了 L2 归一化,将所有特征向量缩放到单位长度。

\subsection{实验参数}
我针对猫和狗的数据分别进行了聚类数等于4的实验。

\section{实验结果与分析}
\subsection{PCA 二维可视化}
为了直观地观察聚类效果,我将高维特征通过主成分分析(PCA)降至二维,并用不同颜色表示每个样本所属的簇。聚类中心也被投影到该二维空间,并用黑色“X”标记。

\begin{figure}[H]
    \centering
    \includegraphics[width=0.7\textwidth]{outputs/wenlan_embeddings_cats_l2_k4_pca2d.png}
    \caption{猫数据集聚类结果的 PCA 二维可视化 ($k=4$)}
    \label{fig:pca_cats}
\end{figure}

\begin{figure}[H]
    \centering
    \includegraphics[width=0.7\textwidth]{outputs/wenlan_embeddings_dogs_l2_k4_pca2d.png}
    \caption{狗数据集聚类结果的 PCA 二维可视化 ($k=4$)}
    \label{fig:pca_dogs}
\end{figure}

\subsection{簇内图像分析}
我还将每个簇内的图像进行拼图展示,便于观察出一些同一簇样本具有的特征。

\begin{figure}[H]
    \centering
    % 使用 2x2 网格展示四个簇的拼图
    \begin{tabular}{cc}
        \includegraphics[width=0.4\textwidth]{outputs/wenlan_embeddings_cats_l2_k4_cluster_01.png} &
        \includegraphics[width=0.4\textwidth]{outputs/wenlan_embeddings_cats_l2_k4_cluster_02.png} \\
        \includegraphics[width=0.33\textwidth]{outputs/wenlan_embeddings_cats_l2_k4_cluster_03.png} &
        \includegraphics[width=0.33\textwidth]{outputs/wenlan_embeddings_cats_l2_k4_cluster_04.png} \\
    \end{tabular}
    \caption{猫数据集的簇内图像}
    \label{fig:cluster_cats_grid}
\end{figure}

\begin{figure}[H]
    \centering
    % 左1右3:设置等高列以保证顶部、底部对齐
    \begin{minipage}[H][0.75\textheight]{0.58\textwidth}
        \centering
        \includegraphics[height=0.75\textheight, keepaspectratio]{outputs/wenlan_embeddings_dogs_l2_k4_cluster_03.png}
    \end{minipage}%
    \hfill
    \begin{minipage}[H][0.6\textheight]{0.38\textwidth}
        \centering
        \includegraphics[width=\textwidth]{outputs/wenlan_embeddings_dogs_l2_k4_cluster_01.png}
        \vfill
        \includegraphics[width=\textwidth]{outputs/wenlan_embeddings_dogs_l2_k4_cluster_02.png}
        \vfill
        \includegraphics[width=\textwidth, height=0.6\textheight, keepaspectratio]{outputs/wenlan_embeddings_dogs_l2_k4_cluster_04.png}
    \end{minipage}
    \caption{狗数据集的簇内图像}
    \label{fig:cluster_dogs_grid}
\end{figure}

观察发现同一簇的猫(狗)具有较为相似的姿态或者毛色特征。
\section{结论}
本实验成功实现并应用了 K-means 算法对图像特征进行聚类。通过 PCA 可视化和簇内图像分析,发现 K-means 能够根据文澜模型提取的视觉特征对图像进行有意义的分组。

\end{document}
