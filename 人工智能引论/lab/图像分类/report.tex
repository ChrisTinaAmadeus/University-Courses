\documentclass[UTF8]{ctexart}
\usepackage{graphicx}
\usepackage{geometry}
\usepackage{float}
\usepackage{amsmath}
\usepackage{titling}

\geometry{a4paper, scale=0.8}
\setlength{\droptitle}{-2cm}
\pagestyle{plain}

\title{图像分类实验报告}
\author{王松宸 2024201594}
\date{\today}

\begin{document}

\maketitle

\section{实验背景与个人理解}
图像分类是计算机视觉中最基础也是最核心的任务之一。其目标是将输入的图像(通常表示为像素矩阵)映射到预定义的类别标签集合中。

在深度学习时代,卷积神经网络(CNN)成为了解决这一任务的主流方法。CNN 通过卷积层提取图像的局部特征(如边缘、纹理),通过池化层降低特征维度并保持平移不变性,最后通过全连接层将提取的高级语义特征映射到类别概率。本次实验中,基于 CIFAR-10 数据集,我设计并实现了模仿 VGG 和 ResNet 架构的两个神经网络模型,并对比了它们的结构特点与运行结果。

\section{网络设计与结果分析}

\subsection{My\_VGG\_Net}

\subsubsection{网络结构设计}
我设计了一个简化版的 VGG 网络(My\_VGG\_Net),其核心设计理念源自 VGG 论文中的“堆叠小卷积核”思想。主要特点如下:

\begin{itemize}
    \item \textbf{Block 结构}:网络由两个卷积块(Block)组成。每个 Block 内部包含多个连续的卷积层,最后接一个最大池化层(Max Pooling)。
    \item \textbf{3x3 卷积与 Padding}:大部分卷积层使用了 $3\times3$ 的卷积核配合 $padding=1$。这样做的好处是卷积操作不改变特征图的尺寸,将尺寸减半的任务完全交给池化层,使得网络结构更加清晰。
    \item \textbf{1x1 卷积}:在第二个 Block 中,我引入了一个 $1\times1$ 的卷积层($padding=0$)。这是模仿 VGG-C 配置的设计,目的是在不改变感受野大小的前提下,增加网络的非线性变换能力,同时调整通道数。
    \item \textbf{全连接层}:最后通过三个全连接层将展平后的特征映射到 10 个分类节点。
\end{itemize}

\subsubsection{网络结构图示}
下表展示了 My\_VGG\_Net 的详细数据流向与张量尺寸变化:

\begin{table}[H]
    \centering
    \begin{tabular}{|c|c|c|c|}
        \hline
        \textbf{阶段} & \textbf{层类型} & \textbf{配置} & \textbf{输出尺寸} \\
        \hline
        输入 & Input & - & $3 \times 32 \times 32$ \\
        \hline
        Block 1 & Conv2d $\times 2$ & $3\times3$, 64 filters & $64 \times 32 \times 32$ \\
        & MaxPool2d & $2\times2$ & $64 \times 16 \times 16$ \\
        \hline
        Block 2 & Conv2d $\times 2$ & $3\times3$, 128 filters & $128 \times 16 \times 16$ \\
        & Conv2d ($1\times1$) & $1\times1$, 128 filters & $128 \times 16 \times 16$ \\
        & MaxPool2d & $2\times2$ & $128 \times 8 \times 8$ \\
        \hline
        分类器 & Flatten & - & 8192 \\
        & Linear & 256 units & 256 \\
        & Linear & 60 units & 60 \\
        & Linear & 10 units & 10 \\
        \hline
\end{tabular}
    \caption{My\_VGG\_Net 网络结构表}
\end{table}

\subsubsection{运行结果与分析}
该网络的运行结果如下图所示:

\begin{figure}[H]
    \centering
    \includegraphics[width=0.8\linewidth]{VGG_Net.png}
    \caption{My\_VGG\_Net 在 CIFAR-10 上的训练与测试结果}
\end{figure}

\textbf{分析}:
VGG 结构通过加深网络深度提取了更丰富的特征。相比于只有两层卷积的 BaselineNet,My\_VGG\_Net 拥有5个卷积层和3个全连接层,具有更强的拟合能力。从结果可以看出,图像分类的准确率提升至74\%。

\subsection{My\_Res\_Net}

\subsubsection{网络结构设计}
我还设计了一个基于残差学习的轻量级网络(My\_Res\_Net),模仿了 ResNet-18 的结构。主要特点如下:

\begin{itemize}
    \item \textbf{残差连接(Shortcut)}:引入了 $F(x) + x$ 的结构。这种跳跃连接允许梯度直接流向浅层,有效缓解了深层网络中的梯度消失问题,使得训练更深的网络成为可能。
    \item \textbf{步长下采样(Stride)}:与 VGG 不同,我在 ResNet 中放弃了最大池化层(Max Pooling),而是在卷积层中设置 $stride=2$ 来进行下采样。这让网络能够自主学习如何压缩特征图,保留更多信息。
    \item \textbf{全局平均池化(GAP)}:在全连接层之前,我使用了 \texttt{AdaptiveAvgPool2d((1, 1))}。这将每个通道的特征图直接压缩为一个数值,极大地减少了全连接层的参数量,降低了过拟合的风险。
\end{itemize}

\subsubsection{网络结构图示}
下表展示了 My\_Res\_Net 的详细数据流向与张量尺寸变化:

\begin{table}[H]
    \centering
    \begin{tabular}{|c|c|c|c|}
        \hline
        \textbf{阶段} & \textbf{层类型} & \textbf{配置} & \textbf{输出尺寸} \\
        \hline
        输入 & Input & - & $3 \times 32 \times 32$ \\
        \hline
        预处理 & Conv2d & $3\times3$, 64 filters & $64 \times 32 \times 32$ \\
        \hline
        Block 1 & ResBlock & $3\times3$, 64 filters & $64 \times 32 \times 32$ \\
        \hline
        Block 2 & ResBlock & $3\times3$, 128 filters, stride 2 & $128 \times 16 \times 16$ \\
        \hline
        Block 3 & ResBlock & $3\times3$, 256 filters, stride 2 & $256 \times 8 \times 8$ \\
        \hline
        分类器 & Global AvgPool & - & $256 \times 1 \times 1$ \\
        & Flatten & - & 256 \\
        & Linear & 10 units & 10 \\
        \hline
    \end{tabular}
    \caption{My\_Res\_Net 网络结构表}
\end{table}

\subsubsection{运行结果与分析}
该网络的运行结果如下图所示:

\begin{figure}[H]
    \centering
    \includegraphics[width=0.8\linewidth]{Res_Net.png}
    \caption{My\_Res\_Net 在 CIFAR-10 上的训练与测试结果}
\end{figure}

\textbf{分析}:
ResNet 的设计使得特征的传递更加顺畅。通过使用 Stride 代替 Pooling,以及引入 GAP,网络能够保持高性能。准确率提升至68\%,低于My\_VGG\_Net的可能原因是Epoch数较少,未完全收敛。但为保证不同网络结构的对比公平性,我未增加训练轮数。

\section{实验收获}
通过本次实验,我不仅熟悉了 PyTorch 框架的基本使用,包括 \texttt{Dataset} 加载、模型定义、损失函数与优化器的选择,更深入理解了 CNN 架构演变背后的逻辑:

\begin{enumerate}
    \item \textbf{架构设计的权衡}:VGG 展示了通过堆叠简单 $3\times3$ 卷积也能达到很好的效果,但参数量较大;ResNet 则通过残差连接解决了深度带来的退化问题。
    \item \textbf{细节的重要性}:在实现 ResNet 时,我深刻体会到了维度匹配的重要性。当特征图尺寸或通道数发生变化时,Shortcut 路径也必须进行相应的 $1\times1$ 卷积变换,否则无法进行张量相加。
    \item \textbf{池化策略}:从 VGG 的 Max Pooling 到 ResNet 的 Stride Convolution + Global Average Pooling,我理解了下采样策略的演进及其对特征保留和参数量的影响。
\end{enumerate}

\end{document}
