\documentclass[UTF8]{ctexart}
\usepackage[margin=1in]{geometry}
\usepackage{graphicx}
\usepackage{caption}
\usepackage{float}
\usepackage{hyperref}
\usepackage{ctex}
\usepackage{titling}

\geometry{a4paper, scale=0.8}
\pagestyle{plain}

\title{语音特征提取实验报告}
\author{王松宸 2024201594}
\date{\today}

\begin{document}
\maketitle

\section{实验目的}
本实验旨在通过使用 Librosa 库,完成音频信号从时域到频域、时频域以及感知域的全面分析。实验将计算并分析幅度包络以理解信号能量随时间的变化,同时,利用快速傅里叶变换(FFT)进行频域分析,绘制幅度谱以观察信号的频率成分。进一步地,通过短时傅里叶变换(STFT)生成语谱图,并结合听觉感知基础绘制 Mel 语谱图。最后,提取梅尔频率倒谱系数(MFCC),掌握这一在语音识别与处理中至关重要的特征提取方法。

\section{Task 1:振幅包络}
振幅包络反映了音频信号能量随时间的轮廓变化,对于检测语音的起始点以及分析信号的动态特性具有重要意义。

在实现上,我将音频信号按照设定的窗口大小(\texttt{frame\_size})进行分帧,并以一定的步长(\texttt{hop\_length})在时间轴上滑动。对于每一个帧,取其信号绝对值的最大值作为该帧的包络值。这种方法能够有效地平滑信号细节,保留整体的能量走势。

下图展示了音频的原始时域波形以及提取出的振幅包络,可以看到包络线很好地描绘了波形的外部轮廓。

\begin{figure}[H]
	\centering
	\includegraphics[width=0.8\textwidth]{figs/waveform.png}
	\caption{音频时域波形}
\end{figure}

\begin{figure}[H]
	\centering
	\includegraphics[width=0.8\textwidth]{figs/amplitude_envelope.png}
	\caption{振幅包络(frame size=1024, hop=512)}
\end{figure}

\section{Task 2:幅度谱}
幅度谱用于展示信号在频率域上的能量分布,帮助我观察信号的主频成分及其谐波结构。

实现过程中,我对整段音频信号执行实数快速傅里叶变换(\texttt{rFFT})。由于实数信号的频谱是共轭对称的,只需关注正频率部分。计算变换结果的幅度,并绘制频率-幅度曲线。

下图为该音频信号的幅度谱(展示前 10\% 的频率范围),从中可以清晰地分辨出信号的主要频率分量。

\begin{figure}[H]
	\centering
	\includegraphics[width=0.8\textwidth]{figs/magnitude_spectrum.png}
	\caption{幅度谱(rFFT,展示 10\% 频率范围)}
\end{figure}

\section{Task 3:语谱图与 Mel 语谱图}
语谱图结合了时域和频域的信息,展示了信号频率随时间的演变过程。

我通过短时傅里叶变换(STFT)实现这一分析:在时间轴上对信号分帧,并对每一帧进行 FFT 变换,从而获得时间-频率-能量的分布矩阵。接着我分别绘制了线性频率刻度和对数分贝刻度的语谱图。

通过对比可以发现,线性频率刻度的语谱图往往难以直观展示低频部分的细节,因为大部分能量集中在低频段,而线性轴将高频部分的空白区域拉得过长,导致关键信息被压缩,视觉上“基本啥也看不到”。相比之下,对数分贝刻度的语谱图通过对能量取对数,显著增强了弱信号的可视性,使得频谱结构更加清晰。

此外,由于人耳对频率的感知是非线性的(对低频更敏感),程序还给出了绘制 Mel 语谱图的代码。它基于 Mel 滤波器组将线性频率映射到感知域,能够更好地模拟人类听觉系统对声音的响应。从结果来看,Mel 语谱图在视觉上最为直观,它剔除了人耳不敏感的高频冗余信息,重点突出了语音信号在低频段的共振峰结构。

\begin{figure}[H]
	\centering
	\includegraphics[width=0.8\textwidth]{figs/spectrogram_linear.png}
	\caption{语谱图(Linear dB)}
\end{figure}

\begin{figure}[H]
	\centering
	\includegraphics[width=0.8\textwidth]{figs/spectrogram_log.png}
	\caption{语谱图(Log dB)}
\end{figure}

\begin{figure}[H]
	\centering
	\includegraphics[width=0.8\textwidth]{figs/melspectrogram.png}
	\caption{Mel 语谱图(Log dB)}
\end{figure}

\section{Task 4:MFCC 特征提取}
梅尔频率倒谱系数(MFCC)是语音识别和音色分析中最经典的特征之一。它在 Mel 能量谱的基础上,通过对数变换和离散余弦变换(DCT)得到。

MFCC 能够提取出频谱的包络信息,去除高频的细节噪声,得到一组低维且稳定的特征系数。

\begin{figure}[H]
	\centering
	\includegraphics[width=0.8\textwidth]{figs/mfccs.png}
	\caption{MFCCs(n\_mfcc=13)}
\end{figure}

\section{总结}
通过本次实验,我深刻体会到了语音信号处理实际上是一个不断提炼核心信息、去除冗余噪声的过程。从最直观的时域波形到频域的幅度谱,再到引入时间维度的语谱图,最后到模拟人耳感知的 Mel 语谱图和 MFCC,每一步变换都有其特定的物理意义和应用价值。

在代码层面,本实验不仅让我初步掌握了 Librosa 库的使用,更建立起了从信号处理理论到实际代码实现的桥梁。
\end{document}
